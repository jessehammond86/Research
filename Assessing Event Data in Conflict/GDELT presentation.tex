%%%%%%%%%%%%%%%%%%%%%%%%%%%%%%%%%%%%%%%%%
% Beamer Presentation
% LaTeX Template
% Version 1.0 (10/11/12)
%
% This template has been downloaded from:
% http://www.LaTeXTemplates.com
%
% License:
% CC BY-NC-SA 3.0 (http://creativecommons.org/licenses/by-nc-sa/3.0/)
%
%%%%%%%%%%%%%%%%%%%%%%%%%%%%%%%%%%%%%%%%%

%----------------------------------------------------------------------------------------
%	PACKAGES AND THEMES
%----------------------------------------------------------------------------------------

\documentclass{beamer}

\mode<presentation> {

% The Beamer class comes with a number of default slide themes
% which change the colors and layouts of slides. Below this is a list
% of all the themes, uncomment each in turn to see what they look like.

%\usetheme{default}
%\usetheme{AnnArbor}
%\usetheme{Antibes}
%\usetheme{Bergen}
%\usetheme{Berkeley}
%\usetheme{Berlin}
%\usetheme{Boadilla}
%\usetheme{CambridgeUS}
%\usetheme{Copenhagen}
%\usetheme{Darmstadt}
%\usetheme{Dresden}
%\usetheme{Frankfurt}
%\usetheme{Goettingen}
%\usetheme{Hannover}
%\usetheme{Ilmenau}
%\usetheme{JuanLesPins}
%\usetheme{Luebeck}
%\usetheme{Madrid}
%\usetheme{Malmoe}
%\usetheme{Marburg}
%\usetheme{Montpellier}
%\usetheme{PaloAlto}
%\usetheme{Pittsburgh}
%\usetheme{Rochester}
\usetheme{Singapore}
%\usetheme{Szeged}
%\usetheme{Warsaw}

% As well as themes, the Beamer class has a number of color themes
% for any slide theme. Uncomment each of these in turn to see how it
% changes the colors of your current slide theme.

%\usecolortheme{albatross}
%\usecolortheme{beaver}
%\usecolortheme{beetle}
%\usecolortheme{crane}
%\usecolortheme{dolphin}
%\usecolortheme{dove}
%\usecolortheme{fly}
%\usecolortheme{lily}
%\usecolortheme{orchid}
%\usecolortheme{rose}
%\usecolortheme{seagull}
%\usecolortheme{seahorse}
%\usecolortheme{whale}
%\usecolortheme{wolverine}

\usefonttheme{professionalfonts}
\usefonttheme{serif} % default family is serif
%\usepackage{fontspec}
%\setmainfont{Liberation Serif}

%\setbeamertemplate{footline} % To remove the footer line in all slides uncomment this line
%\setbeamertemplate{footline}[page number] % To replace the footer line in all slides with a simple slide count uncomment this line

%\setbeamertemplate{navigation symbols}{} % To remove the navigation symbols from the bottom of all slides uncomment this line
}

\usepackage{graphicx} % Allows including images
\usepackage{booktabs} % Allows the use of \toprule, \midrule and \bottomrule in tables

%----------------------------------------------------------------------------------------
%	TITLE PAGE
%----------------------------------------------------------------------------------------

\title[GDELT Analysis]{Machine-Coded Event Data and the Study of Political Violence} % The short title appears at the bottom of every slide, the full title is only on the title page

\author{Jesse Hammond} % Your name
\institute[UC-Davis, Uni-Konstanz] % Your institution as it will appear on the bottom of every slide, may be shorthand to save space
{
University of California, Davis \\ % Your institution for the title page
University of Konstanz\\
\medskip
\textit{jrhammond@ucdavis.edu} % Your email address
}
\date{\today} % Date, can be changed to a custom date

\begin{document}

\begin{frame}
\titlepage % Print the title page as the first slide
\end{frame}

%\begin{frame}
%\frametitle{Overview} % Table of contents slide, comment this block out to remove it
%\tableofcontents % Throughout your presentation, if you choose to use \section{} and \subsection{} commands, these will automatically be printed on this slide as an overview of your presentation
%\end{frame}

%----------------------------------------------------------------------------------------
%	PRESENTATION SLIDES
%----------------------------------------------------------------------------------------

%------------------------------------------------
\section{Outline} 
\subsection{Event Data Discussion} % A subsection can be created just before a set of slides with a common theme to further break down your presentation into chunks
%------------------------------------------------
\begin{frame}
\frametitle{Outline}
\begin{itemize}
\item Brief discussion of event data
\vspace {5 mm}
\item Machine coding as a data-generating tool
\vspace {5 mm}
\item Our project and findings

\end{itemize}
\end{frame}
%------------------------------------------------
\section{Event Data} 
\subsection{Event Data Discussion} % A subsection can be created just before a set of slides with a common theme to further break down your presentation into chunks
%------------------------------------------------

\begin{frame}
\frametitle{What is `Event Data'?}
\begin{itemize}
\item Actors and interactions -- Who did what to whom?
\vspace {5 mm}
\item Recently, location data -- Where did they do it?
\vspace {5 mm}
\item `Event' as a unit of analysis

\end{itemize}
\end{frame}

%------------------------------------------------
\begin{frame}
\frametitle{Why do we care?}
\begin{itemize}
\item Potential for fine-grained analysis of political phenomena
\begin{itemize}
\item Event data deals with \textit{what really happens}
\vspace {2 mm}
\item Less reliance on aggregates and crude coding rules 
\begin{itemize}
\item War = 1,000 deaths?
\end{itemize}
\vspace {2 mm}
\item More data points = more information (probably)
\end{itemize}

\end{itemize}
\end{frame}


%------------------------------------------------
\begin{frame}
\frametitle{Why do we care?}
\begin{itemize}
\item \textcolor{gray}{Potential for fine-grained analysis of political phenomena}
\vspace {2 mm}
\item Event data is increasingly easy to create
\begin{itemize}
\item LexisNexis, Factiva, Google Books/News, etc
\vspace {2 mm}
\item Increasingly powerful tools are available to code data
\end{itemize}

\end{itemize}
\end{frame}



%------------------------------------------------
\begin{frame}
\frametitle{Why do we care?}
\begin{itemize}
\item \textcolor{gray}{Event data is becoming more common}
\vspace {2 mm}
\item \textcolor{gray}{Potential for fine-grained analysis of political phenomena}
\vspace {2 mm}
\item But what do we really know?
\begin{itemize}
\item Event data is extremely high-resolution
\item More data points = more information (probably)
\end{itemize}
\end{itemize}
\end{frame}

%------------------------------------------------

\begin{frame}
\frametitle{What is `Event Data'?}
\begin{itemize}
\item Actors and interactions -- Who did what to whom?
\vspace {5 mm}
\item Recently, location data -- Where did they do it?
\vspace {5 mm}
\item Unit of analysis

\end{itemize}
\end{frame}

%------------------------------------------------
\section{Research Design}
\subsection{Research Design Outline}
\begin{frame}
\frametitle{Our Approach}
\begin{itemize}
\item A
\vspace {5 mm}
\item B
\vspace {5 mm}
\item C

\end{itemize}
\end{frame}

%------------------------------------------------


\begin{frame}
\Huge{\centerline{Thank You!}}
\end{frame}

%----------------------------------------------------------------------------------------

\end{document} 