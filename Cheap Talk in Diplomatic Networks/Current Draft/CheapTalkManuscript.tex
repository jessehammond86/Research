\documentclass[10pt,]{article}
\usepackage[]{mathpazo}
\usepackage{amssymb,amsmath}
\usepackage{ifxetex,ifluatex}
\usepackage{fixltx2e} % provides \textsubscript
\ifnum 0\ifxetex 1\fi\ifluatex 1\fi=0 % if pdftex
  \usepackage[T1]{fontenc}
  \usepackage[utf8]{inputenc}
\else % if luatex or xelatex
  \ifxetex
    \usepackage{mathspec}
  \else
    \usepackage{fontspec}
  \fi
  \defaultfontfeatures{Ligatures=TeX,Scale=MatchLowercase}
\fi
% use upquote if available, for straight quotes in verbatim environments
\IfFileExists{upquote.sty}{\usepackage{upquote}}{}
% use microtype if available
\IfFileExists{microtype.sty}{%
\usepackage{microtype}
\UseMicrotypeSet[protrusion]{basicmath} % disable protrusion for tt fonts
}{}
\usepackage[margin=1in]{geometry}
\usepackage[unicode=true]{hyperref}
\hypersetup{
            pdftitle={Cheap Talk, Costly Action: Using Verbal Interaction to Predict Conflict and Cooperation},
            pdfborder={0 0 0},
            breaklinks=true}
\urlstyle{same}  % don't use monospace font for urls
\usepackage{natbib}
\bibliographystyle{apsr}
\usepackage{longtable,booktabs}
% Fix footnotes in tables (requires footnote package)
\IfFileExists{footnote.sty}{\usepackage{footnote}\makesavenoteenv{long table}}{}
\usepackage{graphicx,grffile}
\makeatletter
\def\maxwidth{\ifdim\Gin@nat@width>\linewidth\linewidth\else\Gin@nat@width\fi}
\def\maxheight{\ifdim\Gin@nat@height>\textheight\textheight\else\Gin@nat@height\fi}
\makeatother
% Scale images if necessary, so that they will not overflow the page
% margins by default, and it is still possible to overwrite the defaults
% using explicit options in \includegraphics[width, height, ...]{}
\setkeys{Gin}{width=\maxwidth,height=\maxheight,keepaspectratio}
\IfFileExists{parskip.sty}{%
\usepackage{parskip}
}{% else
\setlength{\parindent}{0pt}
\setlength{\parskip}{6pt plus 2pt minus 1pt}
}
\setlength{\emergencystretch}{3em}  % prevent overfull lines
\providecommand{\tightlist}{%
  \setlength{\itemsep}{0pt}\setlength{\parskip}{0pt}}
\setcounter{secnumdepth}{5}
% Redefines (sub)paragraphs to behave more like sections
\ifx\paragraph\undefined\else
\let\oldparagraph\paragraph
\renewcommand{\paragraph}[1]{\oldparagraph{#1}\mbox{}}
\fi
\ifx\subparagraph\undefined\else
\let\oldsubparagraph\subparagraph
\renewcommand{\subparagraph}[1]{\oldsubparagraph{#1}\mbox{}}
\fi

% set default figure placement to htbp
\makeatletter
\def\fps@figure{htbp}
\makeatother

\linespread{1.3}


% You know, for landscape
\usepackage{lscape}
\usepackage{pdfpages}


% pandoc does not parse latex env - https://groups.google.com/forum/?fromgroups=#!topic/pandoc-discuss/oZETB5Ii1Cw
\newcommand{\blandscape}{\begin{landscape}}
\newcommand{\elandscape}{\end{landscape}}

% Make new page before each section
\let\stdsection\section
\renewcommand\section{\newpage\stdsection}

% Highlight inline `code`
\usepackage{soul}
\usepackage{xcolor}

\definecolor{Light}{gray}{.97}
\sethlcolor{Light}

\let\OldTexttt\texttt
\renewcommand{\texttt}[1]{\OldTexttt{\hl{#1}}}


\clubpenalty=10000      %kara za sierotki
\widowpenalty=10000  % nie pozostawiaj wdów
\brokenpenalty=10000    % nie dziel wyrazów miêdzy stronami
\exhyphenpenalty=999999   % nie dziel s³ów z myœlnikiem
\righthyphenmin=3     % dziel minimum 3 litery

\renewcommand{\topfraction}{0.95}
\renewcommand{\bottomfraction}{0.95}
\renewcommand{\textfraction}{0.05}
\renewcommand{\floatpagefraction}{0.35}


% Stuff I added.
% --------------

\usepackage{indentfirst}
\usepackage[doublespacing]{setspace}
\usepackage{fancyhdr}
\pagestyle{fancy}
\usepackage{layout}   
\usepackage{caption}
\lhead{\sc Cheap Talk, Costly Action: Using Verbal Interaction to Predict Conflict
and Cooperation}
\chead{}
\rhead{\thepage}
\lfoot{}
\cfoot{}
\rfoot{}

\usepackage{tabularx}
\usepackage{colortbl}
\usepackage{hhline}
%
%\newcommand\MyBox[1]{%
%	\fbox{\parbox[c][1.7cm][c]{1.7cm}{\centering #1}}%
%}
%\newcommand\MyVBox[1]{%
%	\parbox[c][1.7cm][c]{1cm}{\centering\bfseries #1}%
%}  
%\newcommand\MyHBox[2][\dimexpr1.7cm+2\fboxsep\relax]{%
%	\parbox[c][1cm][c]{#1}{\centering\bfseries #2}%
%}  
%\newcommand\MyTBox[4]{%
%	\MyVBox{#1}\MyBox{#2}\hspace*{-\fboxrule}%
%	\MyBox{#3}\hspace*{-\fboxrule}%
%	\MyBox{#4}\par\vspace{-\fboxrule}
%}  

\renewcommand{\headrulewidth}{0.0pt}
\renewcommand{\footrulewidth}{0.0pt}

\usepackage{sectsty}
\sectionfont{\centering}
\subsectionfont{\centering}

\newtheorem{hypothesis}{Hypothesis}

% Begin document
% --------------

\begin{document}

\doublespacing


\begin{center}Abstract\end{center}

\noindent Quantitative tests of constructivist theory tend to rely on structural
dyadic variables such as shared political institutions, societal
similarity, and joint IGO membership to explain conflict and cooperation
between states. In this article, I show that we can improve these
explanations by measuring time-varying levels of affinity or enmity
between states more directly. To measure these underlying relationships,
I analyze verbal interactions between states: threats, promises,
accusations, and so on. These are low-cost, low-information events that
individually say little about how states view one another. In the
aggregate, however, verbal interactions serve two purposes. First,
verbal interactions are imperfect but useful indicators of the
underlying level of trust or affinity between states. Second, over the
long term these interactions constitute identities, socializing states
toward norms of cooperation or conflict with one another. I find that
even after controlling for a range of variables common in the conflict
literature, verbal cooperation and conflict predict material cooperation
and conflict. States that engage one another positively tend to avoid
militarized disputes and are more likely to assist one another during
domestic crises. Conversely, states which engage one another negatively
are more likely to engage in MIDs and more likely to attempt to
destabilize one another during domestic crisis.




\newpage

Word count: 12,006 excluding appendix and abstract.

``Whoever can be trusted with very little can also be trusted with much,
and whoever is dishonest with very little will also be dishonest with
much'' (Luke 16:10, NIV Translation).

\section{Introduction}\label{introduction}

Quantitative international relations scholarship has focused
overwhelmingly on recording and explaining major changes in interstate
behavior such as armed conflict, formation and breaking of alliances,
and waxing or waning of trade flows. This makes sense: given limited
resources and time, scholars focus their efforts on explaining outcomes
that affect the lives and wellbeing of thousands or millions of people.
The limitation of this approach, however, is that it produces a
theoretical and empirical landscape limited to only high-intensity
interactions. Theories and data alike are tailored to explaining the
highest `peaks' and deepest `valleys' over time in international
relations, and have little to say about everything that goes on in
between.

Although they are generally dismissed as `cheap talk', the level and
tone of low-cost, day-to-day verbal interactions between states can be
useful in predicting when and how high-cost forms of cooperation and
conflict occur. This is because these low-cost events make up the vast
majority of \emph{de facto} foreign policy ---- they set the tone of
interstate relations, and over time, aid in building mutually friendly
or oppositional identities. States that cooperate on low-intensity
issues and frame one another as friends, or at least non-threats, are
able to build trust over time. States that trust each other are more not
just more likely to avoid major conflict; they are more likely to
actively cooperate on important economic and security issues.
Conversely, states that view one another with distrust or acrimony are
not only less likely to actively cooperate, but more likely to let
disputes escalate to the level of militarized conflict.

\section{Literature review and
theory}\label{literature-review-and-theory}

This analysis is rooted in constructivist literature focusing on
normative or identity-based factors explaining cooperation and conflict
between states. With minor variations, scholars of the Kantian peace
point out the importance of a triad of factors --- shared democratic
institutions
\citetext{\citealp[e.g.][1158]{Doyle1986a}; \citealp[626]{Russett1993}; \citealp[398-400]{Danilovic2007}},
economic interdependence \citep[e.g.][287-288]{Oneal2003}, and
international organizations and law \citep[e.g.][442]{Dorussen2008} ---
in helping states avoid military crises, and in preventing them from
escalating to all-out war when they do occur. Although it is not without
its critics
\citetext{\citealp[e.g.][174-175]{Gartzke2013}; \citealp[584-585]{Ward2007}}
the finding that stable democracies who share strong economic ties and
membership in the same international organizations are less likely to
engage in militarized conflict with one another is one of the more
prominent findings in the last half-century of political science
\citep[e.g.][388-398]{Oneal2003}.

However, the Kantian peace literature is necessarily limited, by its
very nature, to explaining only a subset of international relations.
Because it emphasizes the importance of democratic institutions (in many
cases, more so than economic ties or international law) the theory has
less to say about when and why peaceful relations emerge between
non-democratic dyads, whether they are mutual autocracies (as a very
general term) or mixed democratic/non-democratic pairs. Peaceful
relations and cooperation may not be as common in these cases as they
are in democratic dyads, but `oases' of peaceful interaction in space
and time exist and are of interest to international relations scholars
\citep[6]{Kivimaki2001}.

Constructivist scholarship presents a generalization of the Kantian
peace, emphasizing the normative and identity-based factors that bind
state dyads together and pointing out that shared norms, ideas, and
identities are not unique to democratic dyads, nor does the general
mechanism of `norms' always lead to cooperation
\citetext{\citealp[e.g.][863]{Ruggie1998}; \citealp[402-404]{Finnemore2001}}.
Work in this tradition attempts to identify when, where, and how shared
\emph{cooperative} identities and norms can emerge across democratic,
non-democratic, and mixed-regime dyads. Constructivist scholars
generally agree that institutional and economic factors matter, but
emphasize a broader role for cooperative norms, ideas, and identities in
promoting peace between states \citep[229]{Wendt1999}. States that trust
one another are more likely to cooperate with one another and less
likely to engage in conflict because they have acquired some level of
``\ldots{} mutual responsiveness, that is, they may gain the ability to
more or less predict one another's behavior and come to \emph{know} each
other as trustworthy'' \citep[254]{Adler1997a}. Shared material or
political characteristics may increase the probability that states will
view each other as similar or trustworthy, but are themselves neither
necessary nor sufficient conditions for cooperation
\citep[19]{Kivimaki2001}. Rather, these similarities lower barriers to
communication and trust, making it more likely that states will
recognize -- and over time, reinforce -- some level of shared identity
or kinship with one another
\citetext{\citealp[1158]{Doyle1986a}; \citealp[862-863]{Ruggie1998}}.

This literature emphasizes that it is \emph{how states view one
another}, not just empirical factors like resources or institutions,
that governs patterns of conflict and cooperation: states that perceive
one another as fundamentally similar are less likely to engage in
conflict \citep[416-417]{Peceny1997}. By contrast, states that define
their identities as mutually conflictual or alien tend not just to lack
common ground for communication and cooperation, but actually infer
oppositional motivations when explaining and assessing the actions of
their partners \citep[502-504]{RisseKappen1995}. This tendency for
states to `believe the worst' about one another makes cooperation much
more difficult, and raises the likelihood of escalation or even
pre-emptive conflict as a form of self-defense against a dangerous and
unreasonable opponent. Identities and ideas are key drivers of state
behavior that often, but not always, correlate with materialistic or
objective similarities in regime type, geographic or cultural proximity,
or economic interdependence. The emergence of shared identities is
possible outside of the democratic-dyad case, even if it less common
\citep[18-19]{Peceny2002}.

Communication plays a crucial role in the emergence of shared norms and
identities. Repeated positive interactions can lead to a fundamental
restructuring of identities: a blurring of the lines between `us' and
`them' that are taken to be exogenous and unchanging in positivist
theories of international relations
\citetext{\citealp[246]{Wendt1999}; \citealp[573-575]{Ashizawa2008}}.
Actions and identities are mutually constitutive: ``\ldots{}even
identities are generated in party by interaction -- both the generic
identities of states qua states and their specific identities''
\citep[879]{Ruggie1998}. Cooperative interactions between states serve
as self-reinforcing evidence of an underlying shared identity; as this
sense of collective identity is shared by both societies, cooperation
and coordination in foreign policy becomes easier and more attractive.
Repeated interactions also provide information about one another's
incentives and preferences:

\begin{quote}
Social acts thus help generate expectations among states about each
other's behavior in future interactions, with each subsequent social act
potentially reinforcing (`reproducing') or modifying these meanings and
expectations. If actions are repeated frequently and consistently enough
(that is, if they are `recursive'), their products become
institutionalized. This makes them appear to be relatively stable and
external elements of objective social reality, and generates a certain
degree of path dependence \citep[104]{Kahl1998}.
\end{quote}

The `social acts' Kahl identifies are rarely operationalized in
empirical work. Instead, tests of the constructivist peace often utilize
structural variables identifying similarities between states
\citep[373-374]{Corbetta2013}. Like other social entities, states that
are closer along economic, political, or socio-cultural dimensions are
more `alike'; they should therefore be able to understand one another
better, should communicate more effectively, and should ultimately be
able to find common ground for cooperation more easily than states that
do not share these characteristics
\citetext{\citealp[27-29]{Blau1977}; \citealp[717-719]{Popielarz1995}}.

However, states with shared characteristics do not automatically
cooperate with one another \citep[495]{RisseKappen1995}. The assumption
that shared characteristics necessarily lead to affinity can be
misleading; many states with long histories of violence share cultural
practices, ethnic demographics, and other important characteristics; in
some cases, shared cultural characteristics are actually associated with
an increase in the likelihood of conflict between states
\citep[77-78]{Gartzke2006}. Using shared characteristics as a measure of
shared identity, therefore, can be something of a pitfall. These
structural analyses are insightful and valuable; however, they
underestimate the importance of interaction and communication, not just
underlying similarity, in building shared identity.

Given the theoretical emphasis on interactions rather than shared
characteristics as the driving force behind identity formation, why does
the extant literature focus primarily on structure? I argue that there
are two reasons that this mechanism has seen little empirical
examination. First, there has been relatively little effort in both the
theoretical and data-gathering realms towards identifying acts of
international \emph{cooperation} which are essential to measuring
friendly international relations. Scholars in conflict studies tend to
treat peace as simply the absence of armed conflict. However, a more
stable peace should logically involve active cooperative engagement,
rather than simply a lack of war. Instead, scholarly efforts have
largely focused on highly salient acts of conflict such as militarized
disputes. This makes sense, given the limited resources available to
scholars and the substantive importance of avoiding tragedy and loss of
life. The result, however, is that we have a much more detailed theories
and data sets describing international conflict than we do international
cooperation. If conflict and cooperation are opposite ends of some
relational spectrum, as \citet[118-120]{Ward1982} suggests, then better
measures of international cooperation may help us better understand and
avoid international conflict.

The second factor limiting our ability to measure and explain
trust-building between states is the lack of attention paid to
low-intensity, routine cooperative and conflictual interactions between
states\footnote{One notable exception to this trend is the work of
  \citet{Jungblut2002} linking the \emph{issue area} of low-intensity
  conflict, using the COPDAB events data set, to the probability of
  militarized interstate dispute. However, they limit their analysis to
  only conflictual event, ignoring acts of interstate cooperation.}.
Most of the outcomes we are interested in as scholars are, naturally,
high-intensity events that involve changes in state power, human
casualties, large increases in economic indicators, and so on. However,
these outcomes are relatively rare compared to the vast majority of
interactions that make up day-to-day international relations. Their
rarity, and their severity, make these events hard to predict in
isolation. Military conflicts, treaty agreements, and other important
acts of conflict and cooperation do not occur in isolation, and they do
not appear suddenly. Rather, these events are usually the culmination of
months or years of low-intensity interactions as states negotiate,
posture, and otherwise assess whether or not they view one another more
as partners or as threats.

States interact near-constantly through diplomatic personnel, government
and military delegations, and other forms of political interaction.
These run-of-the-mill incidents usually go unrecorded because they are
generally not salient. It is not particularly interesting, in isolation,
that the United Kingdom government reassurances of friendship to Saudi
Arabian leadership \citep{BBC2016} or that Chad and Sudan held a meeting
to discuss border security \citep{Tribune2016}. These individual events
do not decisively impact the economic or security environment of the
states involved, and as such, they tend to be treated as unimportant.
When taken as part of a whole, however, we can use these events to infer
underlying patterns of trust and enmity between states.

A pair of states that often engages in low-intensity forms of
cooperation with one another such as symbolic gestures, statements of
support, collaboration on disaster relief or scientific research and so
on, is engaging in trust-building behavior. These routine actions do not
\emph{individually} shape interstate relations, but over time, they
signal and reinforce shared norms and identities of cooperation.
Populations that observe their government engaging in positive relations
with another state are more likely to internalize these norms, leading
to a virtuous cycle in which state leadership and domestic populations
believe that their counterparts in other states are fundamentally
trustworthy or friendly. The result is a normalization of cooperation
between these states.

Positive interactions over time both signal and create shared identity
between states. This not only makes cooperation easier, broadly
speaking, but also lowers barries for states to engage in higher-cost
forms of cooperation that provide greater mutual benefit, but are also
more costly and risky for the states involved. High-intensity
cooperation takes such forms as as joining formal international
institutions; relaxing barriers to travel or trade flows; or delegating
some level of national security to another actor through a defense pact
or military alliance. These policy decisions involve significant risk:
by engaging in these forms of cooperation, states may open themselves up
to defection and the loss of wealth or security, as has been pointed out
repeatedly in functionalist studies of interstate bargaining
\citep[e.g.][271-273]{Fearon1998b}. States choose their partners
carefully, and are more likely to engage in high-intensity cooperation
with partners with which they share some sense of identity or trust;
previous experience serves as a foundation for future cooperation.

Conversely, negative interactions both signal and create a sense of
distrust in which each state views its partner as enemy or alien. States
with acrimonious relations identify one another as opponents, making
cooperation more difficult and conflict easier. This is also a
self-reinforcing cycle: as states define themselves in opposition to one
another, verbal and material conflict becomes the order of the day in
how these states interact. This also means that higher-cost or
higher-risk forms of conflict become a more viable option to solve
disputes or divide stakes. If an opponent cannot be trusted or
understood, it is difficult to identify any shared ground on which an
agreement can be structured, and little reason to expect such an
agreement to be kept.

High-intensity forms of conflict are those that involve significant
political, economic, or human cost. War is the clearest example of
high-intensity conflict, as it can involve major costs up to the
existence of the state itself. Another example is the use conomic
sanctions, a common tool of coercion that exacts costs on both the
initiator and the target. States can also engage rivals indirectly, for
example by giving material or financial aid to insurgent or terrorist
groups operating within another country. In all such cases, states are
willingly engaging in costly or risky behavior in order to either (1)
gain some greater stake or (2) exact some (proportionally greater) cost
on another party.

Low-intensity interactions both signal and create relationships between
states -- over time, these repeated interactions define states'
identities vis-à-vis one another, either as partners/friends or as
rivals/enemies. States are more likely to engage in costly,
high-intensity forms of cooperation with other states with whom they
share a friendly relationship; conversely, states that view one another
as untrustworthy or `alien' are unlikely to risk betrayal through
cooperation, and may view high-intensity conflict as the only viable
foreign policy tool in dealing with one another.

\section{Hypotheses}\label{hypotheses}

From this theory linking low-intensity interactions to underlying bonds
of trust or distrust, I derive several testable hypotheses. The key
predictor in testing these hypotheses is the level of cooperation or
conflict present in low-intensity interactions between states over time.
First, I expect that states that cooperate more on a day-to-day basis
are less likely to experience high-intensity forms of international
conflict. States that share norms of trust and cooperation are able to
engage in peaceful forms of dispute resolution; those that view one
another with acrimony are less able and willing to negotiate, or place
much faith in the outcome of negotiations
\citep[500-502]{RisseKappen1995}. When disputes occur between states
that distrust one another, they are more likely to escalate in severity.
As such, I expect that by looking at patterns of low-intensity
cooperation and conflict in previous time periods, we can better explain
and predict of militarized conflict in the current time period.

\begin{itemize}
\tightlist
\item
  Hypothesis 1: The greater the level of low-cost verbal cooperation
  (conflict) between states \emph{i} and \emph{j}, the lower (higher)
  the probability of a MID occurring between these two states.
\end{itemize}

Not only are cooperative state dyads less likely to experience
high-intensity conflict, they are also more likely to engage in
high-intensity cooperation. One interesting form of costly interstate
cooperation is the choice to establish formal security agreements, which
I generalize with the term `alliances'. Alliances are credible
commitments between states to refrain from attacking one another, or
support one another in issues of national security. They are not entered
into lightly, as they involve sacrificing some level of autonomy over
national defense by delegating security issues to a party outside state
control \citep[904-905]{Morrow1991}, as well as a credible commitment to
intervening in a partner's conflict should one arise
\citep[317-318]{Smith1998}. Alliances represent significant costs and
risks; as such, they are an excellent example of the type of
high-intensity cooperation that requires some underlying level of trust
or affinity between states.

\begin{itemize}
\tightlist
\item
  Hypothesis 2: The greater the level of low-intensity cooperation
  (conflict) between states \emph{i} and \emph{j}, the higher (lower)
  the probability that these states establish a formal alliance.
\end{itemize}

Finally, I expect that the underlying relationship between states
affects their propensity to intervene directly in one another's domestic
affairs, particularly when one state is experiencing civil conflict.
Intervention in another state's conflict is another example of a
high-intensity, costly foreign policy decision. Intervention does not
just have direct material costs: failed interventions can lead to
domestic backlash \citep[51]{Aydin2010}, or even escalation into
interstate military disputes if the target state objects to being
meddled with \citep[480-482]{Gleditsch2008}.

States that see friends and allies undergoing civil conflict are likely
to offer assistance -- financial, material, or even direct military
involvement -- to the beleaguered government. States that share a
relationship of trust and cooperation are more likely to intervene to
support one another during domestic crises in order to keep a stable
relationship, signal norms of cooperation and support, and set a
precedent for future engagement\footnote{See \citep{Corbetta2013} for an
  example of this logic tested in states' supporting or opposing one
  another during interstate conflict.}. States that distrust one another
are more likely to see civil conflict as an opportunity to impose costs
on a rival: by supporting insurgent movements in another state, the
intervener may hope to weaken its opponent by increasing the intensity,
duration, and costs of civil conflict, or even force a regime change if
the existing government is overthrown
\citetext{\citealp[349-351]{BalchLindsay2008}; \citealp[128]{Lounsbery2016}}.
I expect that states that share an underlying level of trust are more
likely to intervene to support one another, whereas states that share
mutual acrimony are more likely to intervene to undermine one another.

\begin{itemize}
\item
  Hypothesis 3a: Given a state dyad (\emph{i, j}) in which state
  \emph{j} is experiencing civil conflict, the greater the level of
  low-intensity cooperation (conflict) between (\emph{i, j}), the higher
  (lower) the probability that state \emph{i} will intervene to support
  insurgents in state \emph{j}.
\item
  Hypothesis 3b: Given a state dyad (\emph{i, j}) in which state
  \emph{j} is experiencing civil conflict: the greater the level of
  low-intensity cooperation (conflict) between (\emph{i, j}), the higher
  (lower) the probability that state \emph{i} will intervene to support
  the government in state \emph{j}.
\end{itemize}

\section{Research design}\label{research-design}

\subsection{Event data to measure trust and
distrust}\label{event-data-to-measure-trust-and-distrust}

Interactions and identities are mutually constitutive. Positive
interactions lead to the establishment of shared norms and identities
between states, while negative interactions establish norms of animosity
and oppositional identities. This means that --- at least to a limited
extent --- we can make inferences about shared identities by observing
interactions between states. States that share cooperative norms and
identities are more likely not just to avoid military conflict, but to
engage in active forms of cooperation.

Events data provide a promising avenue for operationalizing these
interaction-based theories of identity formation. These are highly
granular data structures in which a single observation corresponds to a
discrete interaction between two parties: ``an occurrence standing out
against the run-of-the-mill course of everyday life, a discrete unit of
behavior that can be pinpointed in time and space''
\citep[21]{Merritt1994}. These data provide the closest thing we have to
`ground-level' metrics of interstate interaction. Although events data
has recently experienced a surge in popularity, scholars have been
codifying and analyzing politicized events for over a century
\citetext{\citealp[19-20]{Merritt1994}; \citealp[548-550]{Schrodt2012}}.

Until relatively recently, the sheer workload of generating event data
made it difficult to build data sets with sufficient geographic,
temporal, or issue-area coverage to engage in large-scale studies of
international relations. In recent years, however, significant advances
in both the availability of raw data (mainly digitized news-media
archives) and computational power have led to a resurgence in the
collection and analysis of political event data in international
relations. Large, well-funded data collection initiatives have sprung up
in in many issue areas. Some
\citep[e.g.][]{Raleigh2010, Sundberg2013, Chojnacki2012} focus primarily
on violence and armed conflict, while others
\citep[e.g.][]{Salehyan2012, Nardulli2011} look at a broader range of
events dealing with social and political unrest. A few ongoing efforts
\citep{OBrien2010, Schrodt2014} take a more `universalist' approach,
coding a wide cross-section of politically relevant events using formal
ontologies of both cooperation and. These universalist data sets are
useful in measuring the tone of low-intensity interactions between
states because they adopt an extremely broad approach to gathering and
coding political interactions. For example, the most common system used
to codify these events data is the CAMEO ontological framework
\citep{Gerner2009}, which includes over 170 categories of interaction
between states. The vast majority of these categories describe
low-intensity verbal interactions: statements of support or
condemnation, diplomatic meetings, offers and inquiries, and other
day-to-day interactions that have little impact, in isolation, on
interstate relations \citep[90]{OBrien2010}.

I argue that these low-cost interactions are exactly the type of `social
acts' \citep[104]{Kahl1998} that constitute relationships of trust or
distrust between states. These are minor, low-cost interactions that,
over time, reinforce identities of friendship or enmity. As such,
examining the way states interact with one another may provide useful
information about the underlying relationships between states. This
information is not directly observable using only structural
characteristics such as ethnic, religious, or linguistic similarity, and
can be used in conjunction with existing variables to explain and
predict high-intensity outcomes of cooperation and conflict.

\subsection{Key independent variable: low-intensity
interactions}\label{key-independent-variable-low-intensity-interactions}

Measures of low-intensity cooperation and conflict were constructed
using raw data from the Integrated Crisis Warning System (ICEWS)
available for a 21-year period from January 1, 1995, through December
31, 2015. ICEWS uses fully automated text-analysis methods to extract
event information from a specified set of news media sources
\citep{OBrien2010}. This automated approach is orders of magnitude
faster and less expensive than human coding, allowing ICEWS to process
and generate a prodigious amount of data each day. News reports are
coded to event data format using the Conflict And Mediation Event
Ontology (CAMEO) framework \citep{Gerner2009}, assigning each event to
one of 256 specific categories, which are organized into 20 aggregated
`root codes'.

Table \ref{tab:RootCodes} lists these root codes\footnote{A table
  describing all event-level codes in ICEWS can be found in the
  Appendix.}. ICEWS include a numeric value termed a `Goldstein Score'
based on the metric introduced by \citep[376]{Goldstein1996}. This
measure quantifies the intensity of each event, measuring the extent to
which an event can be described as `cooperative' or `conflictual'. This
metric ranges from -10 to +10, where cooperative events receive positive
scores and conflictual events receive negative scores. Higher absolute
values indicate more intense cooperation or conflict: highly conflictual
interactions like military violence are scored very low, while highly
cooperative actions like sending humanitarian aid are scored very high.

\newpage

\begin{table}

\caption{\label{tab:RootCodes}CAMEO Root codes and Goldstein scores.}
\centering
\begin{tabular}[t]{rrll}
\toprule
Root Code & Mean Score & Type & Action\\
\midrule
1 & 0.74 & Verbal & Make Public Statement\\
2 & 2.31 & Verbal & Appeal\\
3 & 6.03 & Verbal & Express Intent to Cooperate\\
4 & 3.03 & Verbal & Consult\\
5 & 5.28 & Verbal & Engage in Diplomatic Cooperation\\
\addlinespace
6 & 6.84 & Material & Engage in Material Cooperation\\
7 & 7.60 & Material & Provide Aid\\
8 & 6.80 & Material & Yield\\
9 & -2.00 & Verbal & Investigate\\
10 & -5.00 & Verbal & Demand\\
\addlinespace
11 & -2.00 & Verbal & Disapprove\\
12 & -4.31 & Verbal & Reject\\
13 & -6.23 & Verbal & Threaten\\
14 & -6.88 & Material & Protest\\
15 & -7.20 & Material & Exhibit Military Posture\\
\addlinespace
16 & -6.15 & Verbal & Reduce Relations\\
17 & -6.55 & Material & Coerce\\
18 & -9.35 & Material & Assault\\
19 & -9.79 & Material & Fight\\
20 & -9.93 & Material & Engage in Unconventional Mass Violence\\
\bottomrule
\end{tabular}
\end{table}

The result is a daily snapshot of dyadic international interactions with
information about the substantive nature of each interaction, as well as
quantitative estimates of the degree to which each interaction is
cooperative or conflictual. Table \ref{tab:ICEWSEvents} shows a few
ICEWS entries from early 2015 that illustrate the data structure of
these event records.

\begin{table}

\caption{\label{tab:ICEWSEvents}Sample ICEWS event records.}
\centering
\begin{tabular}[t]{rlllrr}
\toprule
Date & Source State & Target State & Event & Root Code & Goldstein Score\\
\midrule
20150101 & United Kingdom & Pakistan & Consult & 4 & 1\\
20150101 & United States & Syria & Use Aerial Weapons & 19 & -10\\
20150101 & Palestinian Territories & Israel & Bring Lawsuit & 1 & -2\\
20150101 & Israel & Nigeria & Provide aid & 7 & 7\\
\bottomrule
\end{tabular}
\end{table}

My theory revolves around the importance of low-cost interactions.
However, because ICEWS codes such a wide range of events, some ICEWs
event categories overlap with the outcomes I am trying to predict. For
example, while ICEWS does not include an event for `war onset', it does
include event codes for escalations of force, military battles, and
other forms of wartime violence. As such, it may be misleading to use
all categories of ICEWS events.

To avoid this issue of endogeneity, I focus solely on verbal
interactions betweeen states. Verbal interactions are generally viewed
to be a low-information method of communication, often referred to as
`cheap talk' as compared to more salient or costly forms of interaction
\citetext{\citealp[e.g.][274-275]{Fearon1998b}; \citealp[599]{Kydd2003}}.
Making claims, threats, or demands is easy, while carrying out the
actions implied in these statements is much more difficult. However,
even cheap talk is not wholly uninformative, especially when the message
it contains is repeated with little variation over time. States that
habitually engage one another positively are signalling some underlying
affinity: even if each individual message has little impact, over time a
consistent positive tone emerges. Again, I do not claim that cooperative
verbal interactions \emph{cause} states to avoid conflict or engage in
high-cost material cooperation. Instead, I argue that cooperative verbal
interactions signal and (over a longer time period) contribute to an
underlying sense of trust or affinity between states.

The challenge is to use the ICEWS data, which contains a wide range of
verbal interactions, to elicit some measure of the level or intensity of
verbal cooperation and conflict between states over some time period.
One way to do this might be to extract the number of events between each
state dyad. However, this creates serious issues of selection bias.
ICEWS relies primarily on American and European English-language
sources, leading to a distinct pattern in which Western,
English-language states receieve significantly more news coverage
overall than smaller, non-Western, non-English-speaking states. As such,
a simple count of events is likely to be heavily weighted by the
behavior of these over-reported states \citep{Barnett2013}.

Rather than taking the simple count of events between state \emph{i} and
state \emph{j}, I instead take the \emph{share of all events} initiated
by state \emph{i} that target state \emph{j} for each category of
interaction listed in ICEWS. By calculating the percentage of \emph{i}'s
activity that targets state \emph{j} instead of states \emph{k, l,
\ldots{}, n}, I can extract more reliable measures of the extent to
which the \emph{i, j} dyad interacts, given the level of media attention
each state receives.

After taking normalized levels of interaction between all state dyads, I
measure the tone of verbal interaction between states using the
Goldstein scores described previously, where larger positive (negative)
scores indicate higher intensity of cooperation (conflict). I take the
net Goldstein score of all normalized interactions between a given state
dyad (\emph{i, j}) in each year:

\[
Net\;Cooperation_{i,j} = \frac{\sum Goldstein_{i \rightarrow j} + \sum Goldstein_{j \rightarrow i}}{2}
\]

The resulting measure is a continous variable indicating the net
cooperative/conflictual tone of verbal interactions between the \emph{i,
j} dyad. Scores lower than 0 indicate a relationship characterized by
higher levels of verbal conflict such as accusations, demands, or
threats; scores greater than 0 indicate a relationship characterized by
cooperative verbal acts such as offers, promises, and statements of
support. I use this measure as a rough proxy for the underlying tone of
the (\emph{i, j}) relationship: states that have a greater sense of
trust or affinity with one another should express this underlying
relationship fairly consistently over time through their public verbal
statements. States that view one another as threats or rivals, even when
they are not actively confronting one another militarily, should
consistently communicate this hostile relationship in their public
statements as well.

\subsection{Control variables}\label{control-variables}

To test the robustness of the relationship between low-cost verbal
interaction and high-cost material cooperation and conflict, I control
for a range of structural variables that have been repeatedly tested in
the previous literature. The controls I employ are seen commonly in
analyses of the Kantian or democratic peace. If I can identify
statistically significant relationships between my key input and output
variables in the presence of these factors, I will be more confident
that these data add to our ability to predict conflict and cooperation
using constructivist theory.

To control for direct trade ties between states, I include yearly data
on dyadic trade flows \citep[version 3.0]{Barbieri2009}. These data
represent the approximate value of trade exports from state i to state j
in a given year, and are logged to account for the skew present in these
data due to differences in the size of state economies. I expect that
increased levels of trade between states, indicating greater economic
interdependence, to be a negative indicator of high-cost conflict and a
positive indicator of high-cost cooperation. Shared IGO membership is
calculated using data collected by \citet{Pevehouse2002}. These data
measure yearly state-level membership in a set of 495 recorded IGOs from
1816-2005. Using these state-level data, I calculate the overlap in
total IGO membership between each pair of states in each year. In line
with previous literature, I expect that state dyads with higher IGO
overlap are are less likely to engage in high-cost conflict and more
likely to engage in high-cost cooperation.

To capture the effects of shared regime, I include a measure of shared
democracy based on the POLITY IV data set \citep{Marshall2015}, which
attempts to codify state regimes on a -10 (fully autocratic) to +10
(fully democratic) scale. This measure is a binary variable indicating
whether a pair of states are both democracies in a given year, receiving
a 1 if both states have a POLITY IV score of +7 or higher. This captures
the core logic of the Kantian peace, focusing on the presence of joint
democracy in particular. I expect that state dyads with joint democratic
institutions are less likely to engage in high-cost conflict and more
likely to engage in high-cost cooperation.

I also include two structural measures that loosely control for cultural
and material similarities between states. First, I proxy for cultural
similarity by looking at the difference in religious practice between
states. To do so, I use the World Religion dataset version 1.1
\citep{Maoz2013}. Data on the proportion of a state's population that
follows a given religion are gathered at five-year intervals, and
disaggregated to yearly measures using linear interpolation. To measure
the overall level of religious similarity between states, I calculate
the Euclidean distance from state \emph{i} to state \emph{j} in a
seven-dimensional space where each dimension corresponds to a state
population's share of seven major religions: Catholicism, Protestantism,
Sunni Islam, Shia Islam, Buddhism, Hinduism, and Judaism. States that
are closer in this multi-dimensional space have similar religious
distributions in their population, while those that are farther away
have increasingly different religious profiles. I expect that states
with more similar religious practices are less likely to engage in
high-cost conflict, and more likely to engage in high-cost cooperation
\citep{Corbetta2013}.

Finally, I control for the capability ratio between states i and j using
yearly National Material Capabilities data version 4.0
\citep{Singer1987}. This measure loosely captures the material power
differential in a state dyad. While these metrics are not directly tied
to the formation of shared identities, they are important to include
regardless, as material and economic factors still matter. Previous work
has shown somewhat mixed results as to how capability ratios affect the
likelihood of conflict, but the bulk of the literature suggests that
greater power asymmetry tend to decrease the probability of conflict
between a pair of states \citep[e.g.][]{Organski1980}.

\subsection{High-intensity conflict: dyadic
MID}\label{high-intensity-conflict-dyadic-mid}

To analyze the relationship between day-to-day interactions and
militarized disputes, I use the dyadic militarized interstate dispute
(MID) data version 3.0, created by the Correlates of War project
\citep{Ghosn2003}. This data set disaggregates a subset of the standard
MID data set \citep{Palmer2015} between 1993 through 2001 down to
directed initiator-target dyads. I convert these directed
initiator-target records to undirected dyads in which the outcome of
interest for a given dyad-year is the onset (regardless of initiator) of
a MID between states \emph{i} and \emph{j}. The dyadic MID data are the
largest limiting factor in the temporal range of data used to test my
hypotheses, because the overlap between ICEWS and the dyadic MID data is
quite small (1995 --- 2001). The resulting data set identifies 183
dyad-years in which a MID occurred.

\subsection{High-intensity cooperation: alliance
formation}\label{high-intensity-cooperation-alliance-formation}

To analyze the relationship between day-to-day interactions and alliance
formation, I use the interstate alliance data set version 4.0
\citep{Gibler2004}. These data record the presence or absence of a
formal security tie between a pair of states in a given year including
defense pacts, neutrality pacts, non-aggression pacts, and entente
agreements, for the years 1946-2012. For this analysis, I treat all
alliance ties as equivalent. The outcome of interest in this case is
whether or not states \emph{i} and \emph{j} enter into a formal security
agreement in a given year. For the six-year period during which I have a
full set of analysis variables, I observe the formation of 44 new
alliances.

\subsection{High-intensity conflict and cooperation: civil conflict
intervention}\label{high-intensity-conflict-and-cooperation-civil-conflict-intervention}

To analyze the relationship between day-to-day interactions and
third-party intervention in civil conflicts, I use the UCDP External
Support database \citep{Hoghbladh2011}. These data record the presence
and type of external intervention into a given civil conflict in a given
year, along with the identity of the intervener(s). Due to the small
range of data available to analyze here, I treat all forms of
intervention as identical. As more data on my other dyadic variables
becomes available, it would be appropriate to disaggregate by different
forms of external support: committing troops to another state's fight is
a more costly and salient act than providing financial aid, for example.
Here, a given dyad (\emph{i, j}) in which state \emph{j} is experiencing
a civil conflict is coded as 0 if state \emph{i} took no action; +1 if
state \emph{i} intervened to support state \emph{j}; or -1 if state
\emph{i} intervened to support an insurgency in conflict with state
\emph{j}'s government. In the six-year data window available, I observe
80 interventions supporting the embattled state, and 91 interventions
supporting insurgent groups.

\subsection{Data structure: politically relevant
dyad-years}\label{data-structure-politically-relevant-dyad-years}

Dyad-year analyses of international relations are plagued with issues of
artificially inflated sample sizes. This is the `Cambodia-Guatemala
problem': the vast majority of interstate dyads have little reason or
opportunity to directly engage in significant levels of conflict or
cooperation with one another, even in an age of increased globalization
and technological sophistication. Including these non-relevant dyads in
statistical analyses vastly inflates the available sample size, with two
effects. First, the number of zeroes in the data set vastly increases,
distorting the distribution of substantively viable data. Second, due to
the effects of large sample size on statistical models
\citep[2]{Head2015}, this makes it more likely that analysis will
produce statistically significant relationships that have virtually no
substantive impact, making it easy for scholars to misunderstand or
overinflate the importance of their findings.

Previous work
\citetext{\citealp[e.g.][63-65]{Maoz1996}; \citealp[106-107]{Maoz2007}}
suggests that subsetting the range of data to include only `politically
relevant' dyads is both theoretically and empirically more appropriate.
Doing so limits a cross-national dyadic sample to only dyads that are
(1) geographically proximal, allowing them opportunities for
interaction, cooperation, and conflict, or (2) involve at least one
major or regional power with the ability to project economic, military,
and political presence over greater distance. As has been pointed out in
previous literature \citep[e.g.][]{Ward2007} explanations of cooperation
and conflict are not applicable when the states involved simply do not
interact with one another. Meta-analysis of politically relevant dyads
shows that limiting the sample to politically relevant dyads may
introduce slight bias, but ``\ldots{}find little or no evidence that
such error or bias leads to erroneous estimation''
\citep[140-141]{Lemke2001}. For the following analyses, I replicate
\citet[s]{Maoz2007} research design by restricting my sample to only
dyads that are geographically proximal, or that involve at least one
major or regional power as defined by the authors However, as a
robustness check, I also run these analyses on the full state-dyad data
set (\(N\) = 79,255). In all cases, the relevant coefficients are
identical in direction and statistical significance. These results are
available in the Appendix.

The finalized data set includes data on 192 states between 1995 and
2001. After subsetting the data to only include politically relevant
dyads, the total number of observations drops to 5,463 dyad-years.
Summary statistics of these data are shown in Table
\ref{tab:SummaryStats}, and a full correlation matrix can be found in
the Appendix.

\begin{table}[!htbp] \centering 
  \caption{\label{tab:SummaryStats}Summary statistics.} 
  \label{} 
\begin{tabular}{@{\extracolsep{5pt}}lcccc} 
\\[-1.8ex]\hline 
\hline \\[-1.8ex] 
Statistic & \multicolumn{1}{c}{Mean} & \multicolumn{1}{c}{St. Dev.} & \multicolumn{1}{c}{Min} & \multicolumn{1}{c}{Max} \\ 
\hline \\[-1.8ex] 
Positive weight & 0.611 & 0.931 & 0.000 & 8.647 \\ 
Negative weight & 0.690 & 1.775 & 0.000 & 20.157 \\ 
Log (trade) & 3.931 & 2.813 & 0.000 & 12.202 \\ 
IGO overlap & 38.968 & 16.204 & 2 & 105 \\ 
Religious distance & 0.557 & 0.334 & 0.001 & 1.372 \\ 
Joint democracy & 0.167 & 0.373 & 0 & 1 \\ 
Relative capacity & 0.860 & 0.148 & 0.500 & 1.000 \\ 
Alliance formation & 0.010 & 0.100 & 0 & 1 \\ 
MID onset & 0.032 & 0.177 & 0 & 1 \\ 
Intervention (state) & 0.013 & 0.113 & 0 & 1 \\ 
Intervention (rebel) & 0.015 & 0.120 & 0 & 1 \\ 
\hline \\[-1.8ex] 
\end{tabular} 
\end{table}

This data set is unfortunately limited due to data availability along
both input and output variables. As Figure \ref{fig:DataTimespan} shows,
the overlap between all variables used in this analysis covers a
six-year period between 1995 and 2001, with the key bottleneck being the
limited overlap between the ICEWS data (which only goes back to 1995)
and the dyadic MID data (which only goes forward to 2001). This is a
limited sample, making me cautious about over-generalizing any findings.
At the very least, however, this sample is a useful test case that
motivates future data-gathering.

\begin{figure}[htbp]
\centering
\includegraphics{CheapTalkManuscript_files/figure-latex/DataTimespan-1.pdf}
\caption{\label{fig:DataTimespan}Timespan of available data.}
\end{figure}

\subsection{Modeling approach}\label{modeling-approach}

To test these hypothesis, I rely on three different variants of logistic
regression analysis. These data are essentially
time-series-cross-sectional (TSCS) observations with discrete outcome
variables (0/1 for MID onset and alliance formation, -1/0/+1 for civil
conflict intervention). As such, two issues have to be dealt with.
First, the temporal nature of these data means that endogeneity is a
potential issue for inferring causal effects. For example, if a MID is
observed in the same year that two states share a high number of
low-cost conflictual interactions, it is not clear which preceded which
--- the MID may be an escalation of low-cost conflict, or the onset of a
MID may be followed by further acrimonious interactions such as claiming
victory or placing blame for initiation. Second, TSCS data do not
fulfill the assumption of independence --- the characteristics of a
given dyad (i, j) are highly correlated over time.

I deal with endogeneity in the crude but common fashion of lagging all
independent variables by one year. This removes the possibility that a
correlation between input and output variables is a result of the output
`causing' the input. It also raises the bar of difficulty for testing my
hypotheses, in that it implicitly assumes that the relationships
signaled by low-cost verbal interactions are stable over time ---- in
other words, that last year's cooperation and conflict on minor issues
captures an underlying relationship that affects the likelihood of
cooperation and conflict on major issues this year.

I also include yearly fixed effects to control for the effects of time.
\citet[1283-1284]{Beck1998} point out that including non-parametric
temporal effects in the form of splines are an effective approach for
TSCS data. Given the six-year temporal span of my data, however, splines
and other transformations are unlikely to be helpful. Instead, I include
yearly fixed effects to allow the probability of any given outcome to
vary as a result of time\footnote{I recognize that at least for modeling
  MID onset, a hazard model is empirically and theoretically more
  appropriate \citep[see][ for a detailed
  discussion]{Box-Steffensmeir2004}. I use logistic regression for two
  reasons. First, while hazard analysis is more appropriate for MIDs, it
  is more difficult to apply and interpret for outcomes subject to
  selection effects (alliances) or for interrupted or irregular
  observation windows (civil conflict interventions). To make it easier
  to compare model results, I use different variants of logistic
  regression for all three tests. Second, logistic regression with
  temporal controls generally produces results that are very similar to
  hazard analysis, reducing the worry that I am biasing or mis-modeling
  these processes. As a robustness check, however, I run a set of Cox
  proportional-hazard models on the MIDs data, and show the results ---
  which are identical in direction and significance --- in the Appendix.}.

\subsubsection{Modeling approach: dyadic
MID}\label{modeling-approach-dyadic-mid}

To model the likelihood of a MID in a given year as a function of
low-intensity cooperation and conflict in the previous year, I employ
logistic regression with fixed effects for time.

\subsubsection{Modeling approach: alliance
formation}\label{modeling-approach-alliance-formation}

Modeling alliance formation is a bit more complicated for two related
reasons. First, alliances tend to be durable over time. This means that
alliances are more akin to phase changes than the discrete/repeatable
events we usually model with logistic regression. Second, the window of
observation is small and historically recent. Taken together, this means
that many of the state dyads most likely to form both trust-based
relationships and alliances already have done so over the previous
decades or centuries ---- which in turn means that they are unlikely to
form new alliances during the window of analysis I have available.

To deal with this issue, I treat the prior presence of an alliance as a
selection effect. To model the likelihood of two states joining in an
alliance, I have to first model the probability that they have
\emph{not} done so prior to entering the pool of data being analyzed ---
in other words, identifying state dyads that could potentially form a
new alliance, given that they have not done so already. I do this
through a two-stage Heckman selection model \citep{Heckman1977}: the
first (selection) stage estimates the likelihood that two states
\emph{do not} have an alliance prior to 1995, and the second (outcome)
stage estimates the probability that two states will form a new
alliance, given that they have not already.

The functional form of the first (selection) stage is a probit
regression identifying state dyads that do not already have an alliance.
Dyads with no alliance present are selected into the second stage of the
model, which uses logistic regression to estimate the probability that
they will form a new alliance.

\subsubsection{Modeling approach: civil conflict
intervention}\label{modeling-approach-civil-conflict-intervention}

To model the likelihood of a given state \emph{i} intervening in another
state \emph{j}'s ongoing civil conflict, I employ multinomial logistic
regression. This allows me to estimate the probability of a given action
by state \emph{i} (support government or support insurgents) relative to
the baseline outcome of doing nothing. For this analysis, I restrict the
sample to only dyad-years in which state \emph{j} is experiencing an
ongoing civil conflict: this is because intervening in another state's
conflict is (obviously) impossible when that state is not experiencing a
conflict to begin with. The resulting data set consists of 908
politically relevant dyad-years in which at least one state in a dyad is
experiencing a civil conflict.

\subsubsection{Testing robustness: out-of-sample
prediction}\label{testing-robustness-out-of-sample-prediction}

Statistical analysis in political science comes under criticism for two
reasons. First, coefficient estimates are often presented as `important'
simply because they meet some standard of statistical significance,
usually \(p < 0.05\), with little discussion of \emph{substantive}
significance: whether realistic changes in those variables would
actually lead to significantly different predictions
\citep[2]{Head2015}. Second, and often related, is the accusation that
statistical models over-fit the data, producing estimates that are
well-suited for explaining existing patterns but virtually useless for
predicting or explaining new data \citep[597]{Ward2007}.

To deal with these issues, I engage in out-of-sample (OOS) testing in
each set of models. OOS modeling trains a model on a subset of the data,
holding a portion `untouched' during the training process. Then, the
trained model is used to make predictions on the untouched data, and
these predictions are compared with the observed outcomes. This approach
lets me gauge whether the estimates produced are robust to over-fitting
and are potentially useful for future predictions.

If the variables I propose are empirically useful, then they will (1)
show statistically significant correlations with the outcome of interest
in the training stage of the model, and (2) improve the predictive power
of the model when faced with new data. Each model is trained on a random
subset of 70\% of observations, with 30\% held back for testing.

\section{Results and discussion}\label{results-and-discussion}

\subsection{Results: dyadic MID onset}\label{results-dyadic-mid-onset}

Hypothesis 1 is strongly supported. I find that there is a statistically
significant, positive relationship between low-cost verbal interactions
between states and the onset of a MID. Table \ref{tab:MidTable} shows
the results of a set of logistic regression models. Model 1 provides a
baseline, including a set of functionalist variables identified as
important in the majority of IR literature. Model 2 introduces a set of
structural controls identified in the Kantian peace literature. Model 3
introduces the event-based measure of net verbal cooperation.

\begin{table}[!htbp] \centering 
  \caption{\label{tab:MidTable}Results: dyadic MID onset.} 
  \label{} 
\begin{tabular}{@{\extracolsep{5pt}}lccc} 
\\[-1.8ex]\hline 
\hline \\[-1.8ex] 
 & \multicolumn{3}{c}{\textit{Dependent variable:}} \\ 
\cline{2-4} 
\\[-1.8ex] & \multicolumn{3}{c}{Dyadic MID onset} \\ 
 & 1 & 2 & 3 \\ 
\hline \\[-1.8ex] 
 Net cooperation &  &  & $-$0.26$^{***}$ \\ 
  &  &  & (0.03) \\ 
  & & & \\ 
 Alliance & 0.48$^{**}$ & 0.65$^{***}$ & 0.53$^{**}$ \\ 
  & (0.21) & (0.23) & (0.24) \\ 
  & & & \\ 
 Logged trade & $-$0.08$^{**}$ & 0.01 & 0.02 \\ 
  & (0.04) & (0.04) & (0.04) \\ 
  & & & \\ 
 IGO overlap &  & $-$0.02$^{**}$ & $-$0.02$^{*}$ \\ 
  &  & (0.01) & (0.01) \\ 
  & & & \\ 
 Religious dist. &  & $-$0.39 & $-$0.57$^{*}$ \\ 
  &  & (0.29) & (0.30) \\ 
  & & & \\ 
 Joint democracy &  & $-$1.24$^{**}$ & $-$1.23$^{**}$ \\ 
  &  & (0.49) & (0.50) \\ 
  & & & \\ 
 Relative capacity & $-$2.35$^{***}$ & $-$2.77$^{***}$ & $-$2.77$^{***}$ \\ 
  & (0.61) & (0.65) & (0.66) \\ 
  & & & \\ 
 Previous MIDs & 1.36$^{***}$ & 1.32$^{***}$ & 1.19$^{***}$ \\ 
  & (0.12) & (0.12) & (0.13) \\ 
  & & & \\ 
 Constant & $-$1.18$^{**}$ & $-$0.21 & $-$0.28 \\ 
  & (0.54) & (0.70) & (0.70) \\ 
  & & & \\ 
\hline \\[-1.8ex] 
Observations & 3,801 & 3,801 & 3,801 \\ 
Log Likelihood & $-$474.11 & $-$465.68 & $-$440.76 \\ 
Akaike Inf. Crit. & 968.23 & 957.37 & 909.52 \\ 
\hline 
\hline \\[-1.8ex] 
\textit{Note:}  & \multicolumn{3}{r}{$^{*}$p$<$0.1; $^{**}$p$<$0.05; $^{***}$p$<$0.01} \\ 
 & \multicolumn{3}{r}{Yearly fixed effects not shown.} \\ 
\end{tabular} 
\end{table}

Model 3 shows that increased levels of verbal cooperation in a given
year are associated with a lower chance of MID onset in the following
year. This effect is both substantively and statistically significant.
Calculating predicted probabilities quantifies this effect more
directly: moving from the 5\% quantile to the 95\% quantile along the
net-cooperation variable, holding all other factors constant, decreases
the likelihood of MID onset by about 33\% (0.03 to 0.02) in a given
year. While a difference of 1\% is not very large, consider that (1)
MIDs are extremely rare events even between politically relevant dyads,
and (2) the effects of verbal cooperation are both substantively and
statistically more significant than those of dyadic trade, IGO
membership, and religious distance. This is a strong indication that the
measure of net verbal cooperation I create is capturing real variation.

It is worth noting that, at least for the sample analyzed here, few of
the control variables suggested by the literature attain statistical
significance after measures of low-cost verbal interaction are included.
The presence of an alliance, levels of dyadic trade, overlapping IGO
membership, and religious similarity all show inconclusive relationships
with MID onset. Joint democracy, relative capacity, and the presence of
previous militarized disputes all remain significant, and in the
directions expected by the literature: dyads that share democratic
institutions are less likely to experience a MID, while those that have
a history of conflict or have similar material capability are more
likely to experience a MID.

For the training data, I find that net cooperation plays an important
role in predicting MID onset. However, it is also important to assess
whether this measure still improves explanatory power when faced with
new data. To do so, I use Models 1 through 3 to predict MID onset using
the 30\% of observations held back from the training step. Testing model
fit on these untouched data shows that, for both in-sample and
out-of-sample data, including measures of low-cost verbal cooperation
and conflict increases model fit when predicting MID onset\footnote{More
  recent work suggests that the effects of both alliances and trade ties
  may be more complex than this dyadic research design might suggest
  \citep[e.g.][]{Oneal2003, Maoz2007}. Future work could expand on this
  dyadic approach, embedding states in networks of interaction that may
  pick up on indirect dependencies.}. To compare how well each model did
in predicting the outcome of interest on the testing data, I measure the
area under the ROC curve (AUC), a commonly used measure of model fit
that assesses how well a model balances between true-positive and
false-positive predictions \citep{Hanley1982}. Values closer to 1
indicate better model fit, while values closer to 0.5 signal a model on
par with random guessing. Model 2, which omits low-intensity
interactions, scores 0.82 on in-sample and 0.77 on out-of-sample
prediction. The fully specified Model 6 improves on this significantly,
scoring 0.86 on in-sample and 0.81 on out-of-sample prediction. Not only
does the fully-specified model fit the training data better, but it
significantly improves predictive accuracy when used on new data. This
provides strong support for the validity of using these data to predict
militarized interstate disputes.

\subsection{Results: alliance
formation}\label{results-alliance-formation}

Hypothesis 2, on the other hand, finds little support. The Heckman
selection models presented in Tables \ref{tab:AllyTableSelect} and
\ref{tab:AllyTableOutcome} predict as a first stage whether two states
did \emph{not} share an alliance in 1995, the first year in the data
pool. The second stage of the model predicts whether two states, given
that they did not share an alliance previously, form a new alliance in a
given year. In this second stage, net verbal cooperation has no
significant effect on alliance formation.

\newpage

\begin{table}[!htbp] \centering 
  \caption{\label{tab:AllyTableSelect}Selection stage: absence of alliance.} 
  \label{} 
\begin{tabular}{@{\extracolsep{5pt}}lccc} 
\\[-1.8ex]\hline 
\hline \\[-1.8ex] 
 & \multicolumn{3}{c}{\textit{Dependent variable:}} \\ 
\cline{2-4} 
\\[-1.8ex] & \multicolumn{3}{c}{Absence of alliance} \\ 
\hline \\[-1.8ex] 
 Logged trade & $-$0.17$^{***}$ & $-$0.05$^{***}$ & $-$0.05$^{***}$ \\ 
  & (0.01) & (0.01) & (0.01) \\ 
  & & & \\ 
 IGO overlap &  & $-$0.05$^{***}$ & $-$0.05$^{***}$ \\ 
  &  & (0.002) & (0.002) \\ 
  & & & \\ 
 Religious dist. &  & 0.79$^{***}$ & 0.79$^{***}$ \\ 
  &  & (0.08) & (0.08) \\ 
  & & & \\ 
 Joint democracy &  & 0.51$^{***}$ & 0.51$^{***}$ \\ 
  &  & (0.08) & (0.08) \\ 
  & & & \\ 
 Relative capacity & 2.07$^{***}$ & 0.42$^{**}$ & 0.42$^{**}$ \\ 
  & (0.16) & (0.20) & (0.20) \\ 
  & & & \\ 
 Constant & $-$0.26$^{*}$ & 2.15$^{***}$ & 2.15$^{***}$ \\ 
  & (0.14) & (0.21) & (0.21) \\ 
  & & & \\ 
\hline \\[-1.8ex] 
 & 3 & 4 & 5 \\ 
Observations & 3,801 & 3,801 & 3,801 \\ 
Log Likelihood & 1,052.73 & 1,384.29 & 1,385.79 \\ 
$\rho$ & $-$0.001  (0.08) & 0.003  (0.05) & 0.003  (0.05) \\ 
\hline 
\hline \\[-1.8ex] 
\textit{Note:}  & \multicolumn{3}{r}{$^{*}$p$<$0.1; $^{**}$p$<$0.05; $^{***}$p$<$0.01} \\ 
 & \multicolumn{3}{r}{Yearly fixed effects not shown.} \\ 
\end{tabular} 
\end{table}

\newpage

\begin{table}[!htbp] \centering 
  \caption{\label{tab:AllyTableOutcome}Outcome stage: alliance formation.} 
  \label{} 
\begin{tabular}{@{\extracolsep{5pt}}lccc} 
\\[-1.8ex]\hline 
\hline \\[-1.8ex] 
 & \multicolumn{3}{c}{\textit{Dependent variable:}} \\ 
\cline{2-4} 
\\[-1.8ex] & \multicolumn{3}{c}{Alliance formation} \\ 
 & 3 & 4 & 5 \\ 
\hline \\[-1.8ex] 
 Net cooperation &  &  & $-$0.002$^{*}$ \\ 
  &  &  & (0.001) \\ 
  & & & \\ 
 Previous MIDs & 0.005 & 0.004 & 0.002 \\ 
  & (0.005) & (0.005) & (0.005) \\ 
  & & & \\ 
 IGO overlap &  & 0.0004$^{**}$ & 0.0004$^{**}$ \\ 
  &  & (0.0002) & (0.0002) \\ 
  & & & \\ 
 Religious dist. &  & $-$0.02$^{***}$ & $-$0.02$^{***}$ \\ 
  &  & (0.01) & (0.01) \\ 
  & & & \\ 
 Joint democracy &  & 0.01$^{*}$ & 0.01$^{*}$ \\ 
  &  & (0.01) & (0.01) \\ 
  & & & \\ 
 Logged trade & 0.001 & $-$0.001 & $-$0.001 \\ 
  & (0.001) & (0.001) & (0.001) \\ 
  & & & \\ 
 Relative capacity & $-$0.02 & $-$0.01 & $-$0.01 \\ 
  & (0.01) & (0.01) & (0.01) \\ 
  & & & \\ 
 Constant & 0.02 & 0.02 & 0.02 \\ 
  & (0.01) & (0.02) & (0.02) \\ 
  & & & \\ 
\hline \\[-1.8ex] 
Observations & 3,801 & 3,801 & 3,801 \\ 
Log Likelihood & 1,052.73 & 1,384.29 & 1,385.79 \\ 
$\rho$ & $-$0.001  (0.08) & 0.003  (0.05) & 0.003  (0.05) \\ 
\hline 
\hline \\[-1.8ex] 
\textit{Note:}  & \multicolumn{3}{r}{$^{*}$p$<$0.1; $^{**}$p$<$0.05; $^{***}$p$<$0.01} \\ 
 & \multicolumn{3}{r}{Yearly fixed effects not shown.} \\ 
\end{tabular} 
\end{table}

In the selection stage, IGO overlap, trade flows, and religious
proximity have the expected effects: states that share membership in
more IGOs, trade more, and have similar religious populations are more
likely to share an alliance when they enter the sample pool. Oddly,
joint democracy has the opposite effect: the coefficient on joint
democracy is significant and positive, indicating that democratic dyads
were actually slightly \emph{less} likely to share an alliance entering
the sample pool.

In the outcome stage, only IGO overlap and trade flows reach
conventional levels of significance, and both show very small
substantive effects. Dyads that share membership in more IGOs and have
more similar religious populations are slightly more likely to form
alliances, given that they had not already done so by 1995. These
effects are again in the expected direction. Joint democracy has a small
positive effect, but this is significant only at the \(p < 0.10\) level.
Net verbal cooperation has little appreciable effect on the likelihood
of two states forming an alliance: the coefficent sign is negative (the
opposite of the expected direction), fails to reach the \(p < 0.05\)
threshold of significance, and is substantively extremely small.

Unsurprisingly given these issues, including net verbal cooperation does
not appreciably increase out-of-sample predictive capability. Compared
to the structural Model 5, the fully specified Model 6 shows no increase
in model fit as measured in AUC score: both models score 0.84 on
in-sample data and 0.71 on out-of-sample data. This provides further
evidence that, at least in this modeling approach, net verbal
cooperation has little effect on the likelihood of alliance formation.
The lack of support for Hypothesis 2 may be a function of the limited
data available. The number of politically relevant dyads that formed
alliances in this period is very small (n = 44, or about 0.8\% of
cases), making it difficult to form robust predictions. Additionally,
the selection modeling approach may not be enough to properly account
for the fact that most `friendly' state dyads already had alliances
during this time period. With more temporal coverage, it would be
possible to engage in a more intensive test of this hypothesis.

Despite the failure to support Hypothesis 2, a simple \(t\)-test
provides some support for the idea that verbal cooperation and alliances
generally go together. States that are allied share significantly higher
levels of low-intensity cooperation (mean = 1.12) compared to states
that do not share an alliance (mean = 0.73), an increase difference of
over 50\%. This does not speak to the causal effects of alliance and
low-intensity cooperation, but it does suggest that the two forms of
cooperation co-vary with one another. Further data gathering and
analysis may shed more light on the details of this relationship.

\subsection{Results: civil conflict
intervention}\label{results-civil-conflict-intervention}

I find strong support for Hypotheses 3a and 3b, as shown in Table
\ref{tab:InterTable}.

\newpage

\begin{landscape}

\begin{table}[!htbp] \centering 
  \caption{\label{tab:InterTable}Model results: civil conflict intervention.} 
  \label{} 
\begin{tabular}{@{\extracolsep{5pt}}lcccccc} 
\\[-1.8ex]\hline 
\hline \\[-1.8ex] 
 & \multicolumn{6}{c}{\textit{Dependent variable:}} \\ 
\cline{2-7} 
\\[-1.8ex] & Conflict intervention (+1 = state, -1 = rebels) & 1 & -1 & 1 & -1 & 1 \\ 
 & 6 & 6 & 7 & 7 & 8 & 8 \\ 
\hline \\[-1.8ex] 
 Net cooperation &  &  &  &  & $-$0.25$^{***}$ & 0.49$^{***}$ \\ 
  &  &  &  &  & (0.04) & (0.09) \\ 
  & & & & & & \\ 
 Alliance & 0.23 & 1.24$^{*}$ & $-$0.25 & 1.64$^{***}$ & $-$0.54$^{**}$ & 1.43$^{***}$ \\ 
  & (0.59) & (0.65) & (0.24) & (0.28) & (0.26) & (0.29) \\ 
  & & & & & & \\ 
 Logged trade & $-$0.29$^{***}$ & 0.54$^{***}$ & $-$0.22$^{***}$ & 0.19$^{***}$ & $-$0.16$^{***}$ & 0.25$^{***}$ \\ 
  & (0.10) & (0.10) & (0.04) & (0.04) & (0.04) & (0.04) \\ 
  & & & & & & \\ 
 IGO overlap &  &  & 0.04$^{***}$ & 0.19$^{***}$ & 0.04$^{***}$ & 0.18$^{***}$ \\ 
  &  &  & (0.01) & (0.02) & (0.01) & (0.02) \\ 
  & & & & & & \\ 
 Religious dist. &  &  & $-$0.74$^{***}$ & $-$0.51$^{*}$ & $-$0.95$^{***}$ & $-$0.57$^{**}$ \\ 
  &  &  & (0.22) & (0.27) & (0.23) & (0.28) \\ 
  & & & & & & \\ 
 Joint democracy &  &  & $-$14.22$^{***}$ & $-$0.62$^{*}$ & $-$8.17$^{***}$ & $-$0.60$^{*}$ \\ 
  &  &  & (0.0000) & (0.32) & (0.001) & (0.33) \\ 
  & & & & & & \\ 
 Relative capacity & $-$2.71$^{**}$ & 14.21$^{***}$ & $-$2.08$^{***}$ & 24.92$^{***}$ & $-$1.14$^{*}$ & 25.02$^{***}$ \\ 
  & (1.20) & (3.48) & (0.63) & (1.60) & (0.64) & (1.67) \\ 
  & & & & & & \\ 
 Previous MIDs & 0.93$^{***}$ & 0.73 & 1.38$^{***}$ & 1.84$^{***}$ & 1.52$^{***}$ & 2.40$^{***}$ \\ 
  & (0.26) & (0.79) & (0.16) & (0.24) & (0.17) & (0.25) \\ 
  & & & & & & \\ 
 Constant & 0.42 & $-$18.29$^{***}$ & 2.00$^{***}$ & $-$28.20$^{***}$ & 1.14 & $-$29.03$^{***}$ \\ 
  & (1.01) & (3.54) & (0.76) & (1.90) & (0.76) & (2.00) \\ 
  & & & & & & \\ 
\hline \\[-1.8ex] 
Akaike Inf. Crit. & 437.26 & 437.26 & 2,780.91 & 2,780.91 & 2,651.72 & 2,651.72 \\ 
\hline 
\hline \\[-1.8ex] 
\textit{Note:}  & \multicolumn{6}{r}{$^{*}$p$<$0.1; $^{**}$p$<$0.05; $^{***}$p$<$0.01} \\ 
 & \multicolumn{6}{r}{Yearly fixed effects not shown.} \\ 
\end{tabular} 
\end{table} 
\end{landscape}

When a state experiences a civil conflict, the propensity of other
states to act, and the form this action takes, is affected significantly
by the nature of the relationship between the conflict state and the
(potential) intervenor. States that share a cooperative relationship
are, compared to the baseline of `doing nothing', less likely to provide
support to insurgents in an ongoing conflict, and more likely to
intervene to support the government of the conflict state. Verbal
signals of friendship and enmity seem to be backed up consistently by
more costly actions of support or opposition, given the opportunity.

The majority of control variables find support here as well, and in the
directions I would anticipate, with a few interesting exceptions. First,
joint democracies tend to stay out of each others' business: democracies
virtually never intervene to support insurgents, but are not necessarily
more likely to support one another directly during civil conflict.
Second, states with more similar religious populations are more likely
to support insurgents in one another's civil conflicts. This may be due
to irredentist conflicts: groups that are similar in population across
two states, but discriminated against in one, may agitate for
intervention to support their fellows during a religious civil conflict.
Finally, state dyads that share a history of militarized conflict are
more likely to intervene on either the side of the insurgent or the
government.

Having found statistical evidence that verbal interaction indicates the
likelihood and form of conflict intervention, I use the same
in-sample/out-of-sample approach to test whether the inclusion of this
new variable actually improves predictive capability. Since AUC
statistics do not translate readily to multi-outcome models, I instead
present confusion matrices for the structural Model 7 and the
fully-specified Model 8 in Tables \ref{tab:ConfMatStruct} and
\ref{tab:ConfMatFull}. These matrices show the predicted veresus
observed outcomes from each model. Higher values on the diagonal
indicate better model fit, as more observed values are accurately
predicted.

Including measures of low-cost verbal cooperation and conflict improves
model fit on the in-sample data as well as predictive accuracy on the
out-of-sample data. These measures improve accuracy mostly by increasing
the model's ability to correctly identify non-intervention cases while
still accurately classifying cooperative and conflictual intervention,
reducing the number of false-positive predictions where no actual
intervention occurs. The structural model predicts civil conflict
interventions with 39\% accuracy, while the fully specified model
increases predictive accuracy to 45\%. This increases my confidence that
including verbal interactions in models of conflict and cooperation has
real effects in improving our predictive capability.

\newpage

\begin{centering}
\captionof{table}{Confusion matrix, structural model.}
\begin{tabularx}{\textwidth}{c|c c c |}
\label{tab:ConfMatStruct}
 & No action & Support ins. & Support state \\
\hhline{----}
No action & 65 \cellcolor[gray]{.8} & 84 & 54  \\
Support ins. & 0 & 16 \cellcolor[gray]{.8} & 0  \\
Support state & 2 & 2 & 9 \cellcolor[gray]{.8} \\
\hhline{~---}
\end{tabularx}


\captionof{table}{Confusion matrix, full model.}
\begin{tabularx}{\textwidth}{c|c c c |}
\label{tab:ConfMatFull}
 & No action & Support ins. & Support state \\
\hhline{----}
No action & 80 \cellcolor[gray]{.8} & 74 & 49  \\
Support ins. & 0 & 15 \cellcolor[gray]{.8} & 1  \\
Support state & 2 & 2 & 9 \cellcolor[gray]{.8} \\
\hhline{~---}
\end{tabularx}

\end{centering}

\section{Conclusions}\label{conclusions}

Cheap talk -- at least in routine day-to-day interactions -- is a useful
predictor of costly action. Fundamentally, much of what we do as
scholars of international relations boils down to identifying and
explaining `friendship' and `enmity' between states. Events data provide
an interesting way to gauge these relationships and model how they
evolve over time. Repeated low-cost acts of verbal cooperation or
conflict over time can stabilize interstate relationships, normalizing
trust or distrust between a pair of states. I find that the tone of
offers, accusations, promises, and so on between a pair of states is a
useful proxy for the underlying level of trust or distrust in that
relationship. While my current findings are much stronger for military
matters -- MIDs and external support for civil war -- I suspect that the
weakness of my results in explaining alliance formation is largely a
function of low data availability. For MIDs and conflict intervention,
my findings are robust to (1) a set of well-established control
variables; (2) subsetting the input data by politically relevant dyads,
as well as the global sample; and (3) predicting outcomes on held-back
testing data. Overall, my findings suggest two things.

First, these findings support constructivist theory that treats
interaction and identity as mutually constitutive: states that cooperate
in minor day-to-day interactions build positive mutual identities over
time. These underlying cooperative or acrimonious relationships affect
the likelihood of high-cost material forms of conflict and cooperation.
The type and tone of verbal interactions do not have a direct
constraining effect on state behavior in the short run, but rather (1)
signal some underlying level of trust or distrust between states, and
(2) assist in the construction of identities and norms over longer time
periods, strengthening these underlying relationships. These results
provide evidence that integrating short-term events data with
longer-term structural characteristics of states can strengthen our
understanding of how interstate relationships evolve over time.

Second, these findings open up a range of interesting future research
questions, particularly in examining the endogenous effects of formal
institutions. The analysis I show here does not try to explain the
\emph{reasons} that some interstate relationships evolve towards
friendship over decades or centuries, while others evolve towards
enmity. This is largely a limitation of data availability: with a
six-year window of analysis, it is difficult to make claims about how
long-term path dependency emerges over time. However, this is an
interesting question with major implications about the long-term role of
structure versus agency in determining patterns of interstate
cooperation and conflict. With more data, it would be possible --
particularly for newer states -- to gauge how their \emph{de facto}
foreign policy evolves in conjunction with their membership in formal
institutions, their trade ties with other states, their security
relationships, and so on. Event data can help paint a much more complete
picture of how states interact with one another.

Do formal institutions create opportunities for low-intensity
cooperation, or do states that already cooperate form institutions with
one another? Being able to measure the `ground state' of day-to-day,
low-cost interactions between states may provide useful leverage in
analyzing what the causal effects of formal ties such as institutions,
trade, and alliances are in promoting shared ideals, identities, and
friendships between states. It makes sense that there is a mutually
reinforcing cycle in which the existence of structural similarities and
institutional bonds can cause, and be caused by, underlying levels of
friendship and enmity between states. This indicates a need for further
study of the available data, as well as increased data-collection
efforts in the future.

\section{Appendix}\label{appendix}

\subsection{Variable correlation
matrix}\label{variable-correlation-matrix}

\begin{tabular}{l|r|r|r|r|r|r|r}
\hline
  & NetCoop & Alliance & Trade & IGO & ReligDist & JointDem & RelCap\\
\hline
NetCoop & 1.00 & -0.01 & -0.08 & -0.02 & -0.07 & 0.00 & 0.04\\
\hline
Alliance & -0.01 & 1.00 & 0.20 & 0.50 & -0.24 & 0.21 & -0.18\\
\hline
Trade & -0.08 & 0.20 & 1.00 & 0.32 & -0.03 & 0.30 & -0.07\\
\hline
IGO & -0.02 & 0.50 & 0.32 & 1.00 & -0.17 & 0.52 & -0.34\\
\hline
ReligDist & -0.07 & -0.24 & -0.03 & -0.17 & 1.00 & -0.08 & 0.04\\
\hline
JointDem & 0.00 & 0.21 & 0.30 & 0.52 & -0.08 & 1.00 & -0.01\\
\hline
RelCap & 0.04 & -0.18 & -0.07 & -0.34 & 0.04 & -0.01 & 1.00\\
\hline
\end{tabular}

\subsection{Robustness check: logit models, all
dyad-years}\label{robustness-check-logit-models-all-dyad-years}

Presented below are a set of models identical to those in the main
manuscript, but employing all dyad-years instead of only those specified
as `politically relevant'. In all cases, the sign and statistical
significance of the key verbal-cooperation variable is identical to the
models presented previously.

\subsubsection{MID onset}\label{mid-onset}

\begin{table}[!htbp] \centering 
  \caption{\label{tab:MidTable}Dyadic MID onset, all dyads.} 
  \label{} 
\begin{tabular}{@{\extracolsep{5pt}}lccc} 
\\[-1.8ex]\hline 
\hline \\[-1.8ex] 
 & \multicolumn{3}{c}{\textit{Dependent variable:}} \\ 
\cline{2-4} 
\\[-1.8ex] & \multicolumn{3}{c}{Dyadic MID onset} \\ 
 & 1 & 2 & 3 \\ 
\hline \\[-1.8ex] 
 Net cooperation &  &  & $-$0.34$^{***}$ \\ 
  &  &  & (0.04) \\ 
  & & & \\ 
 Alliance & 0.97$^{***}$ & 0.98$^{***}$ & 0.88$^{***}$ \\ 
  & (0.18) & (0.22) & (0.22) \\ 
  & & & \\ 
 Logged trade & 0.18$^{***}$ & 0.28$^{***}$ & 0.25$^{***}$ \\ 
  & (0.03) & (0.03) & (0.03) \\ 
  & & & \\ 
 IGO overlap &  & $-$0.01 & $-$0.01 \\ 
  &  & (0.01) & (0.01) \\ 
  & & & \\ 
 Religious dist. &  & $-$0.89$^{***}$ & $-$0.93$^{***}$ \\ 
  &  & (0.23) & (0.23) \\ 
  & & & \\ 
 Joint democracy &  & $-$2.20$^{***}$ & $-$2.06$^{***}$ \\ 
  &  & (0.44) & (0.44) \\ 
  & & & \\ 
 Relative capacity & $-$1.61$^{***}$ & $-$1.76$^{***}$ & $-$1.92$^{***}$ \\ 
  & (0.49) & (0.49) & (0.50) \\ 
  & & & \\ 
 Previous MIDs & 2.22$^{***}$ & 2.08$^{***}$ & 1.89$^{***}$ \\ 
  & (0.12) & (0.12) & (0.12) \\ 
  & & & \\ 
 Constant & $-$4.90$^{***}$ & $-$4.01$^{***}$ & $-$3.95$^{***}$ \\ 
  & (0.42) & (0.50) & (0.51) \\ 
  & & & \\ 
\hline \\[-1.8ex] 
Observations & 55,453 & 55,453 & 55,453 \\ 
Log Likelihood & $-$1,093.57 & $-$1,061.25 & $-$1,032.60 \\ 
Akaike Inf. Crit. & 2,207.15 & 2,148.50 & 2,093.20 \\ 
\hline 
\hline \\[-1.8ex] 
\textit{Note:}  & \multicolumn{3}{r}{$^{*}$p$<$0.1; $^{**}$p$<$0.05; $^{***}$p$<$0.01} \\ 
 & \multicolumn{3}{r}{Yearly fixed effects not shown.} \\ 
\end{tabular} 
\end{table}

\newpage

\begin{table}[!htbp] \centering 
  \caption{\label{tab:AllySelectionAllDyads}Selection stage: absence of alliance, all dyads.} 
  \label{} 
\begin{tabular}{@{\extracolsep{5pt}}lccc} 
\\[-1.8ex]\hline 
\hline \\[-1.8ex] 
 & \multicolumn{3}{c}{\textit{Dependent variable:}} \\ 
\cline{2-4} 
\\[-1.8ex] & \multicolumn{3}{c}{Absence of alliance} \\ 
\hline \\[-1.8ex] 
 Logged trade & $-$0.14 & 0.01 & 0.01 \\ 
  & (Inf.00) & (Inf.00) & (0.01) \\ 
  & & & \\ 
 IGO overlap &  & $-$0.01 & $-$0.01$^{***}$ \\ 
  &  & (Inf.00) & (0.001) \\ 
  & & & \\ 
 Religious dist. &  & 0.04 & 0.04 \\ 
  &  & (Inf.00) & (0.06) \\ 
  & & & \\ 
 Joint democracy &  & 0.12 & 0.12$^{**}$ \\ 
  &  & (Inf.00) & (0.05) \\ 
  & & & \\ 
 Relative capacity & 0.55 & $-$0.09 & $-$0.10 \\ 
  & (Inf.00) & (Inf.00) & (0.08) \\ 
  & & & \\ 
 Constant & 1.03 & 0.32 & 0.35$^{***}$ \\ 
  & (Inf.00) & (Inf.00) & (0.12) \\ 
  & & & \\ 
\hline \\[-1.8ex] 
Observations & 55,453 & 55,453 & 55,453 \\ 
Log Likelihood & 89,284.21 & 94,474.03 & 94,458.26 \\ 
$\rho$ & 1.00  (Inf.00) & 1.00  (Inf.00) & 1.00 \\ 
\hline 
\hline \\[-1.8ex] 
\textit{Note:}  & \multicolumn{3}{r}{$^{*}$p$<$0.1; $^{**}$p$<$0.05; $^{***}$p$<$0.01} \\ 
 & \multicolumn{3}{r}{Yearly fixed effects not shown.} \\ 
\end{tabular} 
\end{table}

\begin{table}[!htbp] \centering 
  \caption{\label{tab:AllyOutcomeAllDyads}Outcome stage: alliance formation, all dyads.} 
  \label{} 
\begin{tabular}{@{\extracolsep{5pt}}lccc} 
\\[-1.8ex]\hline 
\hline \\[-1.8ex] 
 & \multicolumn{3}{c}{\textit{Dependent variable:}} \\ 
\cline{2-4} 
\\[-1.8ex] & \multicolumn{3}{c}{Alliance formation} \\ 
 & 3 & 4 & 5 \\ 
\hline \\[-1.8ex] 
 Net cooperation &  &  & $-$0.0001 \\ 
  &  &  & (0.0001) \\ 
  & & & \\ 
 Previous MIDs & 0.003 & 0.0000 & 0.0001 \\ 
  & (Inf.00) & (Inf.00) & (0.0003) \\ 
  & & & \\ 
 IGO overlap &  & $-$0.0002 & $-$0.0002$^{***}$ \\ 
  &  & (Inf.00) & (0.0000) \\ 
  & & & \\ 
 Religious dist. &  & 0.001 & 0.001 \\ 
  &  & (Inf.00) & (0.0005) \\ 
  & & & \\ 
 Joint democracy &  & 0.004 & 0.004$^{***}$ \\ 
  &  & (Inf.00) & (0.001) \\ 
  & & & \\ 
 Logged trade & $-$0.001 & 0.0003 & 0.0003$^{***}$ \\ 
  & (Inf.00) & (Inf.00) & (0.0001) \\ 
  & & & \\ 
 Relative capacity & 0.002 & $-$0.003 & $-$0.002$^{**}$ \\ 
  & (Inf.00) & (Inf.00) & (0.001) \\ 
  & & & \\ 
 Constant & $-$0.01 & 0.01 & 0.01$^{***}$ \\ 
  & (Inf.00) & (Inf.00) & (0.001) \\ 
  & & & \\ 
\hline \\[-1.8ex] 
Observations & 55,453 & 55,453 & 55,453 \\ 
Log Likelihood & 89,284.21 & 94,474.03 & 94,458.26 \\ 
$\rho$ & 1.00  (Inf.00) & 1.00  (Inf.00) & 1.00 \\ 
\hline 
\hline \\[-1.8ex] 
\textit{Note:}  & \multicolumn{3}{r}{$^{*}$p$<$0.1; $^{**}$p$<$0.05; $^{***}$p$<$0.01} \\ 
 & \multicolumn{3}{r}{Yearly fixed effects not shown.} \\ 
\end{tabular} 
\end{table}

\newpage

\begin{landscape}

\begin{table}[!htbp] \centering 
  \caption{\label{tab:InterTableAllDyads}Civil conflict intervention, all dyads.} 
  \label{} 
\begin{tabular}{@{\extracolsep{5pt}}lcccccc} 
\\[-1.8ex]\hline 
\hline \\[-1.8ex] 
 & \multicolumn{6}{c}{\textit{Dependent variable:}} \\ 
\cline{2-7} 
\\[-1.8ex] & Conflict intervention (+1 = state, -1 = rebels) & 1 & -1 & 1 & -1 & 1 \\ 
 & 6 & 6 & 7 & 7 & 8 & 8 \\ 
\hline \\[-1.8ex] 
 Net cooperation &  &  &  &  & $-$0.85$^{***}$ & $-$0.04 \\ 
  &  &  &  &  & (0.04) & (0.06) \\ 
  & & & & & & \\ 
 Alliance & 1.95$^{***}$ & 1.88$^{***}$ & 0.79$^{***}$ & 1.13$^{***}$ & $-$0.51$^{*}$ & 1.15$^{***}$ \\ 
  & (0.49) & (0.52) & (0.20) & (0.21) & (0.27) & (0.22) \\ 
  & & & & & & \\ 
 Logged trade & $-$0.04 & 0.53$^{***}$ & $-$0.18$^{***}$ & 0.74$^{***}$ & $-$0.24$^{***}$ & 0.73$^{***}$ \\ 
  & (0.08) & (0.07) & (0.03) & (0.03) & (0.04) & (0.03) \\ 
  & & & & & & \\ 
 IGO overlap &  &  & 0.08$^{***}$ & $-$0.004 & 0.04$^{***}$ & $-$0.01 \\ 
  &  &  & (0.01) & (0.01) & (0.01) & (0.01) \\ 
  & & & & & & \\ 
 Religious dist. &  &  & $-$1.91$^{***}$ & $-$3.77$^{***}$ & $-$2.80$^{***}$ & $-$3.64$^{***}$ \\ 
  &  &  & (0.12) & (0.17) & (0.15) & (0.18) \\ 
  & & & & & & \\ 
 Joint democracy &  &  & $-$8.52$^{***}$ & $-$1.01$^{***}$ & $-$13.44$^{***}$ & $-$0.99$^{***}$ \\ 
  &  &  & (0.0000) & (0.21) & (0.0000) & (0.21) \\ 
  & & & & & & \\ 
 Relative capacity & $-$0.62 & 5.29$^{***}$ & 0.39 & 5.78$^{***}$ & 1.20$^{***}$ & 5.75$^{***}$ \\ 
  & (1.10) & (1.83) & (0.31) & (0.47) & (0.37) & (0.47) \\ 
  & & & & & & \\ 
 Previous MIDs & 2.95$^{***}$ & $-$35.56$^{***}$ & 3.71$^{***}$ & $-$9.53$^{***}$ & 2.98$^{***}$ & $-$16.64$^{***}$ \\ 
  & (0.32) & (0.00) & (0.18) & (0.0000) & (0.18) & (0.0000) \\ 
  & & & & & & \\ 
 Constant & $-$4.66$^{***}$ & $-$10.98$^{***}$ & $-$3.01$^{***}$ & $-$5.77$^{***}$ & $-$2.52$^{***}$ & $-$5.73$^{***}$ \\ 
  & (0.99) & (1.73) & (0.32) & (0.46) & (0.37) & (0.46) \\ 
  & & & & & & \\ 
\hline \\[-1.8ex] 
Akaike Inf. Crit. & 717.85 & 717.85 & 6,266.80 & 6,266.80 & 5,539.39 & 5,539.39 \\ 
\hline 
\hline \\[-1.8ex] 
\textit{Note:}  & \multicolumn{6}{r}{$^{*}$p$<$0.1; $^{**}$p$<$0.05; $^{***}$p$<$0.01} \\ 
 & \multicolumn{6}{r}{Yearly fixed effects not shown.} \\ 
\end{tabular} 
\end{table} 
\end{landscape}

\newpage

\subsection{Robustness check: MID onset, Cox proportional hazard
models}\label{robustness-check-mid-onset-cox-proportional-hazard-models}

Per Footnote 5, I also employ Cox proportional hazard regression to
identify whether a technically more appropriate model changes the
findings, presented in Table \ref{tab:MidCoxPH}. Again, I find identical
results in terms of sign and statistical significance. I find no
evidence to reject the null hypothesis of invariant proportional hazards
for the net-cooperation variable, although I do interact the measure of
previous dyadic MIDs with time.

\begin{table}[!htbp] \centering 
  \caption{\label{tab:MidCoxPH}Results: dyadic MID onset, Cox proportional hazards model.} 
  \label{} 
\begin{tabular}{@{\extracolsep{5pt}}lccc} 
\\[-1.8ex]\hline 
\hline \\[-1.8ex] 
 & \multicolumn{3}{c}{\textit{Dependent variable:}} \\ 
\cline{2-4} 
\\[-1.8ex] & \multicolumn{3}{c}{Dyadic MID onset} \\ 
 & 1 & 2 & 3 \\ 
\hline \\[-1.8ex] 
 Net cooperation &  &  & $-$0.21$^{***}$ \\ 
  &  &  & (0.02) \\ 
  & & & \\ 
 Alliance & 0.44$^{**}$ & 0.54$^{***}$ & 0.45$^{**}$ \\ 
  & (0.18) & (0.20) & (0.20) \\ 
  & & & \\ 
 Logged trade & $-$0.08$^{**}$ & 0.01 & 0.01 \\ 
  & (0.03) & (0.04) & (0.04) \\ 
  & & & \\ 
 IGO overlap &  & $-$0.02$^{**}$ & $-$0.02$^{*}$ \\ 
  &  & (0.01) & (0.01) \\ 
  & & & \\ 
 Religious dist. &  & $-$0.48$^{*}$ & $-$0.51$^{**}$ \\ 
  &  & (0.26) & (0.26) \\ 
  & & & \\ 
 Joint democracy &  & $-$1.56$^{***}$ & $-$1.54$^{***}$ \\ 
  &  & (0.53) & (0.53) \\ 
  & & & \\ 
 Relative capacity & $-$2.38$^{***}$ & $-$2.73$^{***}$ & $-$2.72$^{***}$ \\ 
  & (0.54) & (0.57) & (0.57) \\ 
  & & & \\ 
 Previous MIDs & 390.90$^{***}$ & 382.83$^{***}$ & 259.07$^{*}$ \\ 
  & (144.98) & (146.59) & (146.69) \\ 
  & & & \\ 
 Year & $-$7.18$^{**}$ & $-$7.17$^{**}$ & $-$7.22$^{**}$ \\ 
  & (2.94) & (2.94) & (2.92) \\ 
  & & & \\ 
 Previous MIDs * Year & $-$0.20$^{***}$ & $-$0.19$^{***}$ & $-$0.13$^{*}$ \\ 
  & (0.07) & (0.07) & (0.07) \\ 
  & & & \\ 
\hline \\[-1.8ex] 
Observations & 4,351 & 4,351 & 4,351 \\ 
R$^{2}$ & 0.27 & 0.28 & 0.31 \\ 
$\chi^{2}$ & 520.85$^{***}$ (df = 6) & 544.44$^{***}$ (df = 9) & 596.59$^{***}$ (df = 10) \\ 
\hline 
\hline \\[-1.8ex] 
\textit{Note:}  & \multicolumn{3}{r}{$^{*}$p$<$0.1; $^{**}$p$<$0.05; $^{***}$p$<$0.01} \\ 
\end{tabular} 
\end{table}\newpage

\subsection{Full list of CAMEO codes}\label{full-list-of-cameo-codes}

Table \ref{tab:CameoCodes} presents a full list of all CAMEO event
codes.

\begin{longtable}[t]{rrrl}
\caption{\label{tab:CameoCodes}\label{tab:CAMEOcodes}CAMEO Root codes and Goldstein scores.}\\
\toprule
Root Code & Mean Score & Type & Action\\
\midrule
1 & 10 & 0.0 & Make statement, not specified below\\
1 & 11 & -0.1 & Decline comment\\
1 & 12 & -0.4 & Make pessimistic comment\\
1 & 13 & 0.4 & Make optimistic comment\\
1 & 14 & 0.0 & Consider policy option\\
\addlinespace
1 & 15 & 0.0 & Acknowledge or claim responsibility\\
1 & 16 & 3.4 & Make empathetic comment\\
1 & 17 & 0.0 & Engage in symbolic act\\
1 & 18 & 3.4 & Express accord\\
2 & 20 & 3.0 & Appeal, not specified below\\
\addlinespace
2 & 21 & 3.4 & Appeal for cooperation, not specified below\\
2 & 211 & 3.4 & Appeal for diplomatic cooperation\\
2 & 212 & 3.4 & Appeal for material cooperation\\
2 & 22 & 3.4 & Appeal for policy support\\
2 & 23 & 3.4 & Appeal for aid, not specified below\\
\addlinespace
2 & 231 & 3.4 & Appeal for economic aid\\
2 & 232 & 3.4 & Appeal for military aid\\
2 & 233 & 3.4 & Appeal for humanitarian aid\\
2 & 234 & 3.4 & Appeal for military protection or peacekeeping\\
2 & 24 & -0.3 & Appeal for political reform, not specified below\\
\addlinespace
2 & 241 & -0.3 & Appeal for change in leadership\\
2 & 242 & -0.3 & Appeal for policy change\\
2 & 243 & -0.3 & Appeal for rights\\
2 & 244 & -0.3 & Appeal for change in institutions, regime\\
2 & 25 & -0.3 & Appeal to yield\\
\addlinespace
2 & 26 & 4.0 & Appeal to others to meet or negotiate\\
2 & 27 & 4.0 & Appeal to others to settle dispute\\
2 & 28 & 4.0 & Appeal to others to engage in mediation\\
3 & 30 & 4.0 & Express intent to cooperate, not specified below\\
3 & 31 & 5.2 & Express intent to engage in material cooperation,  not specified below\\
\addlinespace
3 & 311 & 5.2 & Express intent to cooperate economically\\
3 & 312 & 5.2 & Express intent to cooperate militarily\\
3 & 32 & 4.5 & Express intent to provide policy support\\
3 & 33 & 5.2 & Express intent to provide aid, not specified below\\
3 & 331 & 5.2 & Express intent to provide economic aid\\
\addlinespace
3 & 332 & 5.2 & Express intent to provide military aid\\
3 & 333 & 5.2 & Express intent to provide humanitarian aid\\
3 & 334 & 6.0 & Express intent to provide military protection or peacekeeping\\
3 & 34 & 7.0 & Express intent to bring political reform, not specified below\\
3 & 341 & 7.0 & Express intent to change leadership\\
\addlinespace
3 & 342 & 7.0 & Express intent to change policy\\
3 & 343 & 7.0 & Express intent to provide rights\\
3 & 344 & 7.0 & Express intent to change institutions, regime\\
3 & 35 & 7.0 & Express intent to yield, not specified below\\
3 & 351 & 7.0 & Express intent to ease administrative sanctions\\
\addlinespace
3 & 352 & 7.0 & Express intent to stop protests\\
3 & 353 & 7.0 & Express intent to accede to political demands\\
3 & 354 & 7.0 & Express intent to release persons or property\\
3 & 355 & 7.0 & Express intent to ease economic sanctions, boycott, or embargo\\
3 & 356 & 7.0 & Express intent allow international involvement (not mediation)\\
\addlinespace
3 & 357 & 7.0 & Express intent to de-escalate military engagement\\
3 & 36 & 4.0 & Express intent to meet or negotiate\\
3 & 37 & 5.0 & Express intent to settle dispute\\
3 & 38 & 7.0 & Express intent to accept mediation\\
3 & 39 & 5.0 & Express intent to mediate\\
\addlinespace
4 & 40 & 1.0 & Consult, not specified below\\
4 & 41 & 1.0 & Discuss by telephone\\
4 & 42 & 1.9 & Make a visit\\
4 & 43 & 2.8 & Host a visit\\
4 & 44 & 2.5 & Meet at a third location\\
\addlinespace
4 & 45 & 5.0 & Mediate\\
4 & 46 & 7.0 & Engage in negotiation\\
5 & 50 & 3.5 & Engage in diplomatic cooperation, not specified below\\
5 & 51 & 3.4 & Praise or endorse\\
5 & 52 & 3.5 & Defend verbally\\
\addlinespace
5 & 53 & 3.8 & Rally support on behalf of\\
5 & 54 & 6.0 & Grant diplomatic recognition\\
5 & 55 & 7.0 & Apologize\\
5 & 56 & 7.0 & Forgive\\
5 & 57 & 8.0 & Sign formal agreement\\
\addlinespace
6 & 60 & 6.0 & Engage in material cooperation, not spec below\\
6 & 61 & 6.4 & Cooperate economically\\
6 & 62 & 7.4 & Cooperate militarily\\
6 & 63 & 7.4 & Engage in judicial cooperation\\
6 & 64 & 7.0 & Share intelligence or information\\
\addlinespace
7 & 70 & 7.0 & Provide aid, not specified below\\
7 & 71 & 7.4 & Provide economic aid\\
7 & 72 & 8.3 & Provide military aid\\
7 & 73 & 7.4 & Provide humanitarian aid\\
7 & 74 & 8.5 & Provide military protection or peacekeeping\\
\addlinespace
7 & 75 & 7.0 & Grant asylum\\
8 & 80 & 5.0 & Yield, not specified below\\
8 & 81 & 5.0 & Ease administrative sanctions, not specified below\\
8 & 811 & 5.0 & Ease restrictions on freedoms of speech and expression\\
8 & 812 & 5.0 & Ease ban on political parties or politicians\\
\addlinespace
8 & 813 & 5.0 & Ease curfew\\
8 & 814 & 5.0 & Ease state of emergency or martial law\\
8 & 82 & 5.0 & Ease popular protest\\
8 & 83 & 5.0 & Accede to demands for political reform\\
8 & 831 & 5.0 & Accede to demands for change in leadership\\
\addlinespace
8 & 832 & 5.0 & Accede to demands for change in policy\\
8 & 833 & 5.0 & Accede to demands for rights\\
8 & 834 & 5.0 & Accede to demands for change in institutions, regime\\
8 & 84 & 7.0 & Return, release, not specified below\\
8 & 841 & 7.0 & Return, release person(s)\\
\addlinespace
8 & 842 & 7.0 & Return, release property\\
8 & 85 & 7.0 & Ease economic sanctions, boycott, embargo\\
8 & 86 & 9.0 & Allow international involvement\\
8 & 861 & 9.0 & Receive deployment of peacekeepers\\
8 & 862 & 9.0 & Receive inspectors\\
\addlinespace
8 & 863 & 9.0 & Allow delivery of humanitarian aid\\
8 & 87 & 9.0 & De-escalate military engagement\\
8 & 871 & 9.0 & Declare truce, ceasefire\\
8 & 872 & 9.0 & Ease military blocka\\
8 & 873 & 9.0 & Demobilize armed forces\\
\addlinespace
8 & 874 & 10.0 & Retreat or surrender militarily\\
9 & 90 & -2.0 & Investigate, not specified below\\
9 & 91 & -2.0 & Investigate crime, corruption\\
9 & 92 & -2.0 & Investigate human rights abuses\\
9 & 93 & -2.0 & Investigate military action\\
\addlinespace
9 & 94 & -2.0 & Investigate war crimes\\
10 & 100 & -5.0 & Demand, not specified below\\
10 & 101 & -5.0 & Demand information, investigation\\
10 & 102 & -5.0 & Demand policy support\\
10 & 103 & -5.0 & Demand aid, protection, or peacekeeping\\
\addlinespace
10 & 104 & -5.0 & Demand political reform, not specified below\\
10 & 1041 & -5.0 & Demand change in leadership\\
10 & 1042 & -5.0 & Demand policy change\\
10 & 1043 & -5.0 & Demand rights\\
10 & 1044 & -5.0 & Demand change in institutions, regime\\
\addlinespace
10 & 105 & -5.0 & Demand mediation\\
10 & 106 & -5.0 & Demand withdrawal\\
10 & 107 & -5.0 & Demand ceasefire\\
10 & 108 & -5.0 & Demand meeting, negotiation\\
11 & 110 & -2.0 & Disapprove, not specified below\\
\addlinespace
11 & 111 & -2.0 & Criticize or denounce\\
11 & 112 & -2.0 & Accuse, not specified below\\
11 & 1121 & -2.0 & Accuse of crime, corruption\\
11 & 1122 & -2.0 & Accuse of human rights abuses\\
11 & 1123 & -2.0 & Accuse of aggression\\
\addlinespace
11 & 1124 & -2.0 & Accuse of war crimes\\
11 & 1125 & -2.0 & Accuse of espionage, treason\\
11 & 113 & -2.0 & Rally opposition against\\
11 & 114 & -2.0 & Complain officially\\
11 & 115 & -2.0 & Bring lawsuit against\\
\addlinespace
12 & 120 & -4.0 & Reject, not specified below\\
12 & 121 & -4.0 & Reject proposal, not specified below\\
12 & 1211 & -4.0 & Reject ceasefire, withdrawal\\
12 & 1212 & -4.0 & Reject peacekeeping\\
12 & 1213 & -4.0 & Reject settlement\\
\addlinespace
12 & 122 & -4.0 & Reject request for material aid\\
12 & 123 & -4.0 & Reject demands for political reform\\
12 & 1231 & -4.0 & Reject demands for change in leadership\\
12 & 1232 & -4.0 & Reject demands for policy change\\
12 & 1233 & -4.0 & Reject demand for rights\\
\addlinespace
12 & 1234 & -4.0 & Reject demand change in institutions, regime\\
12 & 124 & -5.0 & Reject proposal to meet, discuss, or negotiate\\
12 & 125 & -5.0 & Reject mediation\\
12 & 126 & -5.0 & Defy norms, law\\
12 & 127 & -5.0 & Reject accusation, deny responsibility\\
\addlinespace
12 & 128 & -5.0 & Veto\\
13 & 130 & -4.4 & Threaten, not specified below\\
13 & 131 & -5.8 & Threaten non-force, not specified below\\
13 & 1311 & -5.8 & Threaten to reduce or stop aid\\
13 & 1312 & -5.8 & Threaten to boycott, embargo, or sanction\\
\addlinespace
13 & 1313 & -5.8 & Threaten to reduce or break relations\\
13 & 132 & -5.8 & Threaten with administrative sanctions, not specified below\\
13 & 1321 & -5.8 & Threaten to impose restrictions on freedoms of speech and expression\\
13 & 1322 & -5.8 & Threaten to ban political parties or politicians\\
13 & 1323 & -5.8 & Threaten to impose curfew\\
\addlinespace
13 & 1324 & -5.8 & Threaten to impose state of emergency or martial law\\
13 & 133 & -5.8 & Threaten collective dissent\\
13 & 134 & -5.8 & Threaten to halt negotiations\\
13 & 135 & -5.8 & Threaten to halt mediation\\
13 & 136 & -7.0 & Threaten to expel or withdraw peacekeepers\\
\addlinespace
13 & 137 & -7.0 & Threaten with violent repression\\
13 & 138 & -7.0 & Threaten to use military force, not specified below\\
13 & 1381 & -7.0 & Threaten blockade\\
13 & 1382 & -7.0 & Threaten occupation\\
13 & 1383 & -7.0 & Threaten unconventional violence\\
\addlinespace
13 & 1384 & -7.0 & Threaten conventional attack\\
13 & 1385 & -7.0 & Threaten attack with WMD\\
13 & 139 & -7.0 & Give ultimatum\\
14 & 140 & -6.5 & Engage in popular protest, not specified below\\
14 & 141 & -6.5 & Demonstrate or rally\\
\addlinespace
14 & 1411 & -6.5 & Demonstrate for change in leadership\\
14 & 1412 & -6.5 & Demonstrate for policy change\\
14 & 1413 & -6.5 & Demonstrate for rights\\
14 & 1414 & -6.5 & Demonstrate for change in institutions, regime\\
14 & 142 & -6.5 & Conduct hunger strike, not specified below\\
\addlinespace
14 & 1421 & -6.5 & Conduct hunger strike for change in leadership\\
14 & 1422 & -6.5 & Conduct hunger strike for policy change\\
14 & 1423 & -6.5 & Conduct hunger strike for rights\\
14 & 1424 & -6.5 & Conduct hunger strike for change in institutions, regime\\
14 & 143 & -6.5 & Conduct strike or boycott, not specified below\\
\addlinespace
14 & 1431 & -6.5 & Conduct strike or boycott for change in leadership\\
14 & 1432 & -6.5 & Conduct strike or boycott for policy change\\
14 & 1433 & -6.5 & Conduct strike or boycott for rights\\
14 & 1434 & -6.5 & Conduct strike or boycott for change in institutions, regime\\
14 & 144 & -7.5 & Obstruct passage, block\\
\addlinespace
14 & 1441 & -7.5 & Obstruct passage to demand change in  leadership\\
14 & 1442 & -7.5 & Obstruct passage to demand policy change\\
14 & 1443 & -7.5 & Obstruct passage to demand rights\\
14 & 1444 & -7.5 & Obstruct passage to demand change in institutions, regime\\
14 & 145 & -7.5 & Protest violently, riot\\
\addlinespace
14 & 1451 & -7.5 & Obstruct passage to demand change in  leadership\\
14 & 1452 & -7.5 & Obstruct passage to demand policy change\\
14 & 1453 & -7.5 & Obstruct passage to demand rights\\
14 & 1454 & -7.5 & Obstruct passage to demand change in institutions, regime\\
15 & 150 & -7.2 & Demonstrate military or police power, not specified below\\
\addlinespace
15 & 151 & -7.2 & Increase police alert status\\
15 & 152 & -7.2 & Increase military alert status\\
15 & 153 & -7.2 & Mobilize or increase police power\\
15 & 154 & -7.2 & Mobilize or increase armed forces\\
16 & 160 & -4.0 & Reduce relations, not specified below\\
\addlinespace
16 & 161 & -4.0 & Reduce or break diplomatic relations\\
16 & 162 & -5.6 & Reduce or stop aid, not specified below\\
16 & 1621 & -5.6 & Reduce or stop economic assistance\\
16 & 1622 & -5.6 & Reduce or stop military assistance\\
16 & 1623 & -5.6 & Reduce or stop humanitarian assistance\\
\addlinespace
16 & 163 & -6.5 & Halt negotiations\\
16 & 164 & -7.0 & Expel or withdraw, not specified below\\
16 & 1641 & -7.0 & Expel or withdraw peacekeepers\\
16 & 1642 & -7.0 & Expel or withdraw inspectors, observers\\
16 & 1643 & -7.0 & Expel or withdraw aid agencies\\
\addlinespace
16 & 165 & -7.0 & Halt mediation\\
16 & 166 & -8.0 & Impose embargo, boycott, or sanctions\\
17 & 170 & -7.0 & Coerce, not specified below\\
17 & 171 & -9.2 & Seize or damage property, not specified below\\
17 & 1711 & -9.2 & Confiscate property\\
\addlinespace
17 & 1712 & -9.2 & Destroy property\\
17 & 172 & -5.0 & Impose administrative sanctions, not specified below\\
17 & 1721 & -5.0 & Impose restrictions on freedoms of speech and expression\\
17 & 1722 & -5.0 & Ban political parties or politicians\\
17 & 1723 & -5.0 & Impose curfew\\
\addlinespace
17 & 1724 & -5.0 & Impose state of emergency or martial law\\
17 & 173 & -5.0 & Arrest, detain, or charge with legal action\\
17 & 174 & -5.0 & Expel or deport individuals\\
17 & 175 & -9.0 & Use violent repression\\
18 & 180 & -9.0 & Use unconventional violence, not specified below\\
\addlinespace
18 & 181 & -9.0 & Abduct, hijack, or take hostage\\
18 & 182 & -9.5 & Physically assault, not specified below\\
18 & 1821 & -9.0 & Sexually assault\\
18 & 1822 & -9.0 & Torture\\
18 & 1823 & -10.0 & Kill by physical assault\\
\addlinespace
18 & 183 & -10.0 & Conduct suicide, car, or other non-military bombing, not spec below\\
18 & 1831 & -10.0 & Carry out suicide bombing\\
18 & 1832 & -10.0 & Carry out car bombing\\
18 & 1833 & -10.0 & Carry out roadside bombing\\
18 & 184 & -8.0 & Use as human shield\\
\addlinespace
18 & 185 & -8.0 & Attempt to assassinate\\
18 & 186 & -10.0 & Assassinate\\
19 & 190 & -10.0 & Use conventional military force, not specified below\\
19 & 191 & -9.5 & Impose blockade, restrict movement\\
19 & 192 & -9.5 & Occupy territory\\
\addlinespace
19 & 193 & -10.0 & Fight with small arms and light weapons\\
19 & 194 & -10.0 & Fight with artillery and tanks\\
19 & 195 & -10.0 & Employ aerial weapons\\
19 & 196 & -9.5 & Violate ceasefire\\
20 & 200 & -10.0 & Engage in unconventional mass violence, not specified below\\
\addlinespace
20 & 201 & -9.5 & Engage in mass expulsion\\
20 & 202 & -10.0 & Engage in mass killings\\
20 & 203 & -10.0 & Engage in ethnic cleansing\\
20 & 204 & -10.0 & Use weapons of mass destruction, not specified below\\
20 & 2041 & -10.0 & Use chemical, biological, or radiologicalweapons\\
20 & 2042 & -10.0 & Detonate nuclear weapons\\
\bottomrule
\end{longtable}

\renewcommand\refname{References}
\bibliography{/Users/localadmin/Dropbox/Research/Bibliography/library.bib}

\end{document}

